\section{Data \& Computing Resources}
\label{sec:datacomp}

{\it Necessary resources for data and computing, beyond core LSST project deliverables, to achieve LSST cosmological science goals. Adam Bolton and others.}

The dark-energy science mission of LSST imposes numerous requirements on 
data resources and computing infrastructure for its success. Many of these requirements 
may be beyond the current baseline for LSST construction 
and operations, and some may also be beyond capabilities currently 
anticipated within the DOE-supported LSST Dark-Energy Science Collaboration (DESC)\@.
This section describes these required capabilities as they flow from dark-energy 
science requirements.

\subsection{Data-set access and definition}

The analysis of large LSST datasets for derivation of dark-energy parameter constraints 
will involve development and application of analysis codes by large teams of 
scientists working over periods which are likely to extend over many years, and which will 
require long-term stability and accessibility for scientific reproducibility. This translates 
into requirements that LSST's released static-sky datasets be: 

\begin{itemize}
\item Version controlled
\item Exposed through machine-readable programmatic interfaces
\item Maintained on accessible storage resources for many-year timescales
\end{itemize}

\subsection{Combination of LSST data with other data sets}

Multiple dark-energy science analyses of LSST data will rely on the combination 
of LSST data with redshifts and object parameters from other surveys (both imaging 
and spectroscopic). These analyses impose requirements for: 

\begin{itemize}
\item Long-term access to curated copies of auxiliary datasets
\item Computing infrastructure sufficient for co-analysis of LSST data with auxiliary data
\end{itemize}

\subsection{Likelihood characterization}

Cosmological analyses of LSST data sets will require accurate physical characterization of 
the populations of tracer objects used. This will be necessary both to understand the 
selection-functions of the samples themselves and their evolution with redshift, 
and to connect with physical models for the bias recipes through which these 
populations are related to the underlying dark-matter distribution. In many instances, the 
salient physical parameters of tracers in LSST data may be measured at a level of significance 
that does not admit sufficient characterization through simple Gaussian errors or covariances. 
Such analyses will therefore require: 

\begin{itemize}
\item Explicit identification of all LSST catalog-object 
parameters that are relevant as input to cosmological analysis
\item Sufficient characterization of the joint likelihood of these measured 
parameters on an object-by-object basis, either through direct reporting 
within LSST catalogs, or through custom determination using LSST Level 3 
software infrastructure
\end{itemize}

\subsection{Redshift data validation}

Cosmological analyses of LSST data that rely on spectroscopic redshifts will impose requirements on 
the quality of those redshift data in terms of completeness, purity, precision, and accuracy. In the case 
of spectroscopic samples adopted from other surveys, the requirements for LSST analysis may be 
different than those imposed by the primary science case of those other surveys. Therefore, all 
spectroscopic samples used in LSST dark-energy science analysis will require: 

\begin{itemize}
\item Quantitative specification of the science requirements for redshift and parameter measurements
\item Controlled software infrastructure for validating spectroscopic catalogs 
against LSST dark-energy science requirements
\item Software-development infrastructure for the modification of redshift- and parameter-measurement codes, 
for cases in which a failure to meet requirements is traced not to spectroscopic target-selection or 
observing, but rather to the analysis software used for delivering redshifts from the data
\end{itemize}

\subsection{Mock catalogs}

Simulations, and mock catalogs based on those simulations, will be necessary for multiple 
dark-energy science applications of LSST. These include the characterization of covariances in 
large-scale structure measurements across length scales, determination of the statistics of 
line-of-sight density projection effects for strong-lensing systems, and forward modeling 
of statistical estimators as a function of cosmological parameters. These applications will require: 

\begin{itemize}
\item Allocation of sufficient computing and storage resources to conduct suites of cosmological 
$N$-body simulations tailored to LSST dark-energy science analysis
\item Support for the generation, documentation, and application of mock LSST catalogs based on 
appropriate simulations.
\end{itemize}

\subsection{Survey characterization}

Precision cosmological measurements of the baryon acoustic oscillation (BAO) feature using LSST
data will be sensitive to percent-level systematics that may vary across the LSST footprint in a 
complex field-dependent geometric pattern. In order that these variations do not compromise 
the precision and accuracy of the cosmological measurements, the following capabilities 
will be required: 

\begin{itemize}
\item Sufficient analysis within core LSST observing and data-reduction operations to enable 
determination of field-dependent quantities relevant for dark-energy science analysis 
\item Computation of weight-masks for bright stars, stellar number density, effective seeing, and 
effective depth in LSST static-sky data releases, along with a documented and 
supported data-model for the encoding of these masks
\item Software infrastructure for incorporating these masks into LSST dark-energy parameter analysis
\end{itemize}


