\label{sec:photoz}

{\it Jeffrey Newman, Samuel Schmidt and others}

LSST dark energy constraints, as well as almost all LSST extragalactic science, will be critically dependent upon photometric redshifts (a.k.a. photo-$z$'s): i.e., estimates of the redshifts of objects based only on flux information obtained through broad filters.  In this section, we describe the utilitization of spectroscopy for photometric redshifts for two separate purposes:
\begin{itemize}
\item {\it Training}, that is, making algorithms more effective at predicting the actual redshifts of galaxies, reducing random errors.  Essentially, {\bf the goal of training is to minimize all moments of the distribution of differences between photometric redshift and true redshift}, rendering the correspondence as tight as possible, and hence maximizing the three-dimensional information in a sample; and
\item {\it Calibration}, the determination of the actual bias and scatter in redshifts produced by a given algorithm (for most purposes, this reduces to the problem of determining the actual redshift distribution for samples selected based on some set of properties).  Essentially, {\bf the goal of calibration is to determine with high accuracy the moments of the distribution of true redshifts of the samples used for a given study}.
\end{itemize}
Different datasets will be needed for each of these purposes.  However, as will be described below, the same instrumental capabilities -- and often the same datasets -- needed for photometric redshift training or calibration can also contribute to LSST cosmology in a variety of other ways; we will describe some of these applications at the end of this section.






\subsection{Science Goals}
%
% {\it Describe your science goals here, as in the scientific justification of an NOAO proposal. If you have multiple science goals, you can either describe them all here, or replicate the science, technical, and capabilities sections for each goal. Just create one summary table for the entire program.  }

%NOAO guidelines:
%
%The scientific justification should explain the overall goals of
%your program in the context of your field, as well as the importance
%of your program to astronomy.
%Writing a good scientific justification is an art.  It takes
%skill and practice.  And it requires a good scientific idea.
%This last you must supply but a few general guidelines
%about proposal writing might still be helpful...
%
%\begin{itemize}
%\item
%State succinctly and clearly the problem you are trying to solve
%and the progress that will be made toward doing so if the proposed
%observations are successful.  If the review panel members have to work hard
%even to understand what you want to do, they are unlikely to be
%sympathetic to your proposal.
%
%\item
%Explain clearly why the project is important and how it
%relates to the broad context and important issues in your field.
%Many proposals focus too tightly on a specific observational
%goal (e.g. ``measure the velocity dispersion of this cluster of galaxies'')
%without explaining why it is important or how it relates to a
%significant question about the Universe.
%
%\item
%Be specific.  If your observations will ``constrain theoretical
%models,'' then discuss what will be constrained and why those
%constraints matter.  Make sure the review panel understands exactly why
%the observations you propose will make a difference in your field,
%and exactly how the observations will refine or
%require changes in the theory.
%
%\item
%Keep it simple.  Try to focus on the central idea of your proposal.
%Complex arguments are hard to explain and hard for the panel members to follow.
%Distracting tangential arguments obscure the theme of your proposal.
%
%\item
%Include a figure to help explain what you want to do.  Sample
%data or model predictions shown in a figure often help clarify
%complex arguments for the panel members.
%
%\item
%Keep it short.  Never exceed a page for the text of the scientific
%justification, and never reduce the font size.  It may even help to
%be a little under a page, and increase the font size a little!
%Organize your presentation with paragraphs, headings, and bullets
%so it is easy to read.

\label{sec:photoz_just}

%LSST dark energy constraints, as well as almost all LSST extragalactic science, will be critically dependent upon photometric redshifts (a.k.a. photo-$z$'s): i.e., estimates of the redshifts of objects based only on flux information obtained through broad filters.
%
%Improvements that yield higher-quality, lower-RMS-error photo-$z$'s will result in smaller random errors on parameters of interest and enable analyses in narrower redshift bins; while systematic errors in photometric redshift estimates, if not constrained, may dominate all other uncertainties when studying cosmological parameters.  Both optimizing and calibrating photo-$z$'s are dependent upon obtaining secure redshift measurements from spectroscopy.

%\subsection{\Training}

{\bf Training:} Photo-z methods generally use samples of objects with known z to develop or refine algorithms, and hence to reduce the random errors on individual photometric redshift estimates. This will result in smaller random errors on cosmological parameters of interest and enable analyses in narrower redshift bins.  The larger and more complete the training set is, the smaller the RMS error in photo-$z$ estimates will be, increasing LSST's constraining power.
% For instance, LSST delivers photo-$z$'s with an RMS error of $\sigma_z \sim 0.025(1+z)$ for $i<25.3$ galaxies in simulations with perfect template knowledge and realistic photometric errors, whereas photo-$z$'s in actual samples of similar S/N have delivered $\sigma_z \sim 0.05(1+z)$.
With a perfect training set of galaxy redshifts down to LSST magnitude limits, we could achieve the system-limited performance; this would improve the Dark Energy Task Force figure of merit from LSST lensing+BAO studies by $\sim25$\%, with greater impact in other areas (e.g., studies of galaxy clusters).  {\bf Better photometric redshift training will improve almost all LSST extragalactic science, and hence addresses a wide variety of decadal science goals.}  To enable this, we need secure spectroscopic redshifts for as wide a range of galaxies as possible down to the $i=25.3$ magnitude limit of the LSST weak lensing ``gold sample'').

%\subsection{Calibration}

{\bf Calibration:} Similarly, secure spectroscopic redshifts are needed for {calibration}; e.g., the determination of both any overall bias in photo-$z$'s  as well as their true scatter.  {\bf Inadequate calibration will lead to systematic errors in the use of photo-$z$'s, affecting almost all extragalactic science with LSST and hence many decadal science priorities}.
% It is estimated that the true mean redshift for each sample used for LSST cosmology (e.g., objects selected to be within some bin in photometric redshift) must be known to $\sim 2\times10^{-3}(1+z)$, i.e., 0.2\%.
If extremely high completeness (>99.9\%) is attained in the spectroscopic samples used for training, then LSST calibration requirements would be met directly. However, existing deep redshift samples have failed to obtain secure redshifts for a systematic 20\%-60\% of their targets; it is therefore quite likely that future small-area, deep redshift samples will not solve the calibration problem.

Instead, we can utilize cross-correlation methods to calibrate photo-$z$'s.  These techniques correlate the positions on the sky of objects with known $z$ with the positions of the galaxies whose redshift distribution we aim to characterize.  We can then exploit the fact that bright galaxies trace the same underlying dark matter distribution as fainter objects, enabling the determination of the $z$ distribution for purely photometric samples with high accuracy.  The key data needed for photo-$z$ calibration via cross-correlations are redshift samples with very large numbers of objects over a wide area of sky, spanning the full redshift range of LSST targets of interest.




\subsection{Technical Description }

% {\it Give a technical description of your program, describing e.g., sample size and properties, justification of spectral or spatial resolution, wavelength, target density, etc.}

%NOAO guidelines:
%The review panel looks to this section to find out about the overall
%strategy of your observational program.  Why do you need the telescopes
%and instruments you request? How are your targets selected?
%Why do you need spectroscopy or imaging, and what measurements will
%you make from the data?  Why is your approach to be preferred over
%some other approach, what must the minimum sample size be to achieve
%your scientific goals (and why), and why are your
%observations likely to be better than previous work in the field?

\label{sec:photoz_design}

{\bf Training:} A previous whitepaper on {\it Spectroscopic Needs for Imaging Dark Energy Experiments} \citep{Newman15} has explored in detail the minimum characteristics a photometric redshift training set should have.  We summarize those conclusions here.  In short, we require:

\begin{itemize}
\item {\bf Spectroscopy of at least 30,000 objects down to the survey magnitude limits}, in order to characterize both the core of the photo-$z$/spectroscopic-z relationship and outliers (cf. \citealt{MaHuterer}, \citealt{BernsteinHuterer}, and \citealt{Hearin10}); this will require large exposure times on large telescopes.
\item {\bf High multiplexing}, as obtaining large numbers of spectra down to faint magnitudes will be infeasible otherwise.
\item {\bf Coverage of as broad a wavelength range as possible}, in order to cover multiple spectral features, which is necessary for getting the required {\bf highly-secure} redshifts.  At minimum, spectra should cover from $\sim 4000$ to $\sim 10,000$ Angstroms, but coverage from $\sim 0.3$ to $\sim 1.5\mu$m would be advantageous.
\item {\bf Moderately high resolution ($R>\sim 4000$) at the red end}, critical as it enables secure redshifts to be obtained from the [OII] 3727 Angstrom doublet alone.  High resolution also enables sensitive spectroscopy over the $\sim 90\%$ of the spectrum that lies between the skylines (cf. Newman et al. 2013).
\item {\bf Field diameters $>\sim20$ arcmin}, needed to span multiple correlation lengths to enable accurate clustering measurements.
%This is key for enabling the training samples to be well-understood, providing synergistic galaxy evolution science, and enabling the spectra obtained to be utilized with cross-correlation techniques; it also reduces sample/cosmic variance in the training set.  A typical correlation length of $r_0 \sim 5 h^{-1}$ Mpc comoving corresponds to $\sim 7.5$ arcmin at z=1 and 13 arcmin at z=0.5; hence,
Larger ($>1$ deg) fields would be even better, particularly at low redshifts.
\item And finally, {\bf coverage of as many fields as possible ($\sim 15$ minimum)}, in order to minimize the impact of sample/ cosmic variance.
%Cunha et al. (2012) estimated that 40-150 $\sim$ 0.1 deg$^2$ fields would be needed for DES for sample variance not to impact errors; however, we can do somewhat better by taking advantage of the fact that the variance itself is directly observable via the field-to-field variation in redshift distributions.  Nevertheless,
%with fewer than $\sim 15$ independent (i.e., widely separated) fields, it would be difficult to measure dispersions in counts amongst the fields directly, as measurements of standard deviations are broad and highly skewed for small N. We thus require at least this many fields.
\end{itemize}


{\bf Calibration:} As described in \citet{Newman15},  cross-correlation calibration of photometric redshifts for LSST should require spectroscopy of a minimum of $\sim 5 \times 10^4$ objects over multiple independent $>100$ square degree fields, with coverage of the full redshift range of those objects whose photometric redshifts will be calibrated (for LSST dark energy science, this is essentially $0<z<3$).



\subsection{Needed Capabilities and Estimate of Demand}

% {\it Describe the needed capabilities and demand (e.g., estimate of observing time) that flow down from the science and technical considerations. If applicable, describe the time critical nature of the required capabilities (do you need to have the capabilities while LSST is carrying out the survey or can you do the follow up later?) }

{\bf Training:} In principle, a number of spectrographs currently in existence or being planned have sufficient wavelength coverage and spectral resolution to obtain secure redshifts over a wide range of galaxy properties for photo-z training.  However, the time required will depend on the instrumental and telescope characteristics.  If sky coverage is low, the limiting factor will be how many tilings of the sky will be needed to cover $>15$ fields that are 20 arcmin in diameter.  If multiplexing is low, the limiting factor will be how many tilings are needed to reach 30,000 spectra.  Finally, if both of these factors are high enough that each field need only be observed once, the limiting factor will be how much exposure time it takes the telescope to achieve the desired depth.  Formulae for how exposure time scales in these scenarios are given in \citet{Newman15}.
%
%The actual time required for a survey in each of these scenarios will be:
%
%{\it FOV-limited}:  $t_{observe}$ =(280 nights) $\times$ ($N_{fields}$ / 15) $\times$ (314 arcmin$^2$ / area per pointing )$\times$( Equivalent Keck/DEIMOS exposure time / 100 hours) $\times$ (0.67 / observing efficiency) $\times$ (76 m$^2$  / telescope effective collecting area) $\times$ (0.3 / [telescope + instrument throughput]),
%
%{\it Multiplex-limited}:  $t_{observe}$ =(280 nights) $\times$ ($N_{objects}$ / 3$\times10^4$) $\times$ (2000 / Number of simultaneous spectra)$\times$(Equivalent Keck/DEIMOS exposure time / 100 hours) $\times$ (0.67 / observing efficiency) $\times$ (76 m$^2$  / telescope effective collecting area) $\times$ (0.3 / [telescope + instrument throughput]), or
%
%{\it Fields-limited}:  $t_{observe}$ =(280 nights) $\times$ ($N_{fields}$ / 15) $\times$ (Equivalent Keck/DEIMOS exposure time / 100 hours) $\times$ (0.67 / observing efficiency) $\times$ (76 m$^2$  / telescope effective collecting area) $\times$ (0.3 / [telescope + instrument throughput]),
%
%where $t_{observe}$ is the total clock time required for observations (including overheads and weather, which both reduce observing efficiency); $N_{fields}$ is the total number of fields to be observed; and $N_{objects}$ is the total number of objects to be observed.
%For any given telescope/instrument combination, the actual exposure time required will be the greatest out of the fields-limited, FOV-limited, and multiplex-limited values.
100 hours' Keck/DEIMOS exposure time would be sufficient to achieve the same signal-to-noise for $i=25.3$ objects that DEEP2 obtains at $i \sim 22.5$; at that magnitude, DEEP2 obtained secure redshifts for $\sim 75\%$ of targets.  A spectrograph with broader wavelength range or higher spectral resolution would be expected to do at least as well as this at equivalent signal-to-noise.

\citet{Newman15} tabulates the properties of a variety of current and upcoming spectrographs and estimates the total time they would require to complete the proposed survey.  It would take more than 10 years with Keck/DEIMOS, 5 years with Mayall/DESI,  $\sim1.8$ years with TMT/WFOS, just over 1 year with Subaru/PFS, or as little as 4-5 months on GMT or E-ELT (depending on the final characteristics of instruments whose design is still in flux).  If less telescope time is available, it will be necessary to either allow spectroscopic redshift failure rates to increase or to reduce the depth of the sample; it is likely that the former would have smaller impact on photometric redshift training.  By design, this training sample would also be sufficient to meet LSST calibration requirements {\bf if} spectroscopic redshift failure rates of order $\sim 0.1\%$ could be achieved.  However, based on past experience (with 20-60\% failure rates) we expect to need less direct methods for calibrating photometric redshifts.

We note that the results of this survey -- a set of galaxies spanning the full range of galaxy properties down to the LSST magnitude limit with a maximal amount of spectroscopic information -- will enable a wide variety of galaxy evolution science going well beyond just the training of photometric redshifts.  A number of applications for such a sample are discussed in Chapter XYZ; the sample described in that chapter has considerable overlap, but (if IR spectral coverage is available) somewhat shorter estimated exposure times.  Ideally, this spectroscopy would occur early in the lifetime of LSST, but training redshifts will be useful whenever they are obtained.

{\bf Calibration:} LSST photometric redshift calibration requirements should be met by the overlap between LSST and planned baryon acoustic oscillation experiments.
%Those projects are optimized by targeting the brightest galaxies at a given redshift over the broadest redshift range and total sky area possible; this matches the needs for cross-correlation measurements well.  For photometric redshift calibration, all redshifts used must be highly secure ($>$99\% probability of being correct); however, here incompleteness is not an issue.
For instance, DESI should obtain redshifts of $>\sim$30 million galaxies and QSOs over the redshift range $0 < z < 4$ over more than 14,000 square degrees of sky \citep{DESI}.  It is expected to have $>4000$ square degrees of overlap with LSST.
%Absorption-line systems detected in DESI QSO spectra can provide even more redshifts at low z (Menard et al. 2013).
%We anticipate that DESI will provide several million redshifts within the footprint of LSST.
%As can be seen in Figure XYZ, DESI will allow detailed analyses of photometric redshift calibration, providing multiple cross-checks for systematics.

However, the DESI survey will cover only the northern portion of the LSST sky.  Photometric redshift performance may not be the same there as elsewhere in the LSST footprint (e.g., due to airmass differences), which could yield a miscalibration when applied to LSST as a whole.  This risk can be mitigated by DESI-like spectroscopy in the south.  There are plans to conduct a DESI-like survey with the 4MOST instrument that would fulfill this need well.  If this does not happen, it would be extremely valuable to have access to a DESI-like spectrograph (with wide field of view, $\sim 5000\times$ multiplexing, full optical coverage, sufficient resolution to split [OII], and a $\sim 4m$ diameter telescope aperture) in the Southern hemisphere.  With such an instrument, covering the non-DESI LSST footprint would take comparable observing time to the original DESI survey.

\subsection{Other applications of the required instrumentation}

The same instrumentation needed for a photometric redshift training survey would be well-suited to address a number of other important issues  affecting cosmological measurements with LSST.  We describe a key example here.

{\bf Intrinsic Alignment Studies:} Some of the strongest LSST constraints on dark energy are
expected to come from measurements of the apparent shearing of galaxy images by weak gravitational
lensing.  Characterizing and mitigating systematic uncertainties will be critical to ensure that
LSST weak lensing measurements are not systematics dominated.  The photometric redshift training and
calibration datasets will address one important systematic for weak lensing, but the multi-object
spectroscopy can also help characterize and constrain another systematic: intrinsic alignments (IA)
of galaxy shapes with the cosmic web, which are a contaminant to weak lensing measurements \citep{2015SSRv..193....1J}.

%Several
%methods have been developed for mitigating this systematic, including nulling, forward modeling, and self-calibration.
Existing methods of mitigating this systematic all have limitations, making it important also to
explore intrinsic alignment effects directly, using redshifts to determine
which galaxies are in physical proximity to each other.
%This direct exploration will provide
%intrinsic alignments models as inputs to the mitigation methods that need models.  % JAN: I added that last sentence in response to a question from Joan Najita:
%How critical is the spectroscopy? The first paragraph mentions several methods -- do they all need spectroscopy? If not, how much does the spectroscopy add to the science result? (Is it "nice to have" or "critical"?)
%In particular, we are lacking knowledge of how intrinsic alignments scale with
%galaxy type, luminosity, and redshift at $z> 0.5$.
%However, we need similar information both for
%galaxies at higher redshifts and for blue galaxies in general in order to develop reasonable priors for carrying out the WL analysis with LSST. Generally, this
%requires both good imaging and redshift information, in order to both localize the galaxy pairs in
%3D and measure galaxy shapes.   LSST data itself will ultimately be quite informative about
%IA despite the size of the photo-$z$ errors, but there is still value in
%Large sets of spectroscopic redshifts can provide strong external priors on
%IA models.
% using spectroscopic redshifts, because IA are degenerate with other systematics (photo-$z$ errors and certain shear systematics).
%In principle, a
A direct measurement of IA would require a spectroscopic or spectro-photometric dataset that covers substantial
contiguous areas ($\gtrsim$1 deg$^2$) while sampling a
{\em representative} galaxy sample of tens of thousands of galaxies at minimum.
%As an example of the required field sizes, 1 degree at $z=0.8$ corresponds to a distance of $\sim 35 h^{-1}$ Mpc comoving, which should be well-suited to
%constraining IA models on the $\lesssim 10 h^{-1}$Mpc scales where existing theoretical models are
%most uncertain.  The desirability of large field sizes is somewhat in tension with the approach of surveying many modest-size fields to mitigate sample/cosmic variance as employed in the ``training'' dataset described in
%Sec.~\ref{sec:design}), but l
Larger-field spectrographs such as Subaru/PFS could provide the necessary areal coverage and sampling as part of a photo-z training survey.  If the photo-z training survey covers smaller fields, supplemental spectroscopy with the same sort of instrument required for training work could be necessary.
%it is quite possible that a training set that is still well-suited for IA studies could be developed (we note, for instance, that if the Subaru/PFS spectrograph were used for the training samples, it would have a minimum field size well-suited for IA studies).

Because high redshift precision is not needed, an alternative approach would be to supplement LSST photo-z's with many-band narrow-band imaging or low-resolution spectroscopy; however, high signal-to-noise at faint magnitudes would be necessary, requiring new instrumentation on large telescopes.
%
%An alternative a
%It is worth noting that, although the medium-resolution MOS spectroscopy envisioned for photo-$z$ training could serve the needs of IA characterization, the redshift RMS accuracy needed for this purpose is not particularly high.  Many-band narrow-band imaging, low-resolution spectroscopy, or spectro-photometric redshifts similar to those from the PRIMUS survey could deliver sufficiently low errors, presuming that highly-secure redshifts are obtained from them (the $\sim 5--10\%$ incorrect-z rates from PRIMUS would likely be problematic here). However, the galaxies for which redshifts to characterize IA for LSST are required are much fainter than those studied in previous low-resolution surveys; new instrumentation on larger telescopes would be needed to meet this goal.
Another alternative would be to constrain IA using cross-correlations with spectroscopic galaxies instead of employing
autocorrelation measurements (cf. \citealt{Blazek} and \citealt{Chisari}), which could utilize the same data needed for the photo-$z$ cross-correlation calibration method.
%More exploration is needed to determine precise survey design requirements for this supporting science case, but we anticipate they will be
Having both photo-$z$ training and calibration sets available, and making sure they are
well-optimized for IA purposes, will provide us with multiple potential routes to developing and
testing IA models, maximizing the chances of success.


\subsection{Applications requiring smaller fields of view}

Although the most urgent cosmological needs for multi-object spectroscopy require wide fields of view (of order 0.5 -- 1 degree) to span the typical scales of large-scale structure, cases which focus on more unusual classes of object require dense placement of spectra over smaller fields of view, $\sim 5-10$ arcmin at most (though other requirements are similar to those for wide field spectroscopy).  Because the relevant galaxies are relatively tightly packed on the sky and multiplex requirements are not high, existing or planned slit spectrographs such as Keck/LRIS, Keck/DEIMOS, Gemini/GMOS, and TMT/WFOS could be suitable for this work.  If fibers can be packed tightly enough, a wide-field multiobject spectrograph could also contribute to these activities, but may be less optimal.  We discuss a few example applications of such spectroscopy below. 

{\bf Galaxy Cluster Studies:}
Cosmology with the cluster mass function may become one of the most powerful probes of the cosmology in the LSST era, if cluster masses can be measured to high enough accuracy and precision (Cosmic Visions report; Krause and Eifler 2016).  Moreover, they provide avenues for probing gravity as well as the properties of dark matter.  Multi-object spectroscopy plays a major role on all of these:

\begin{itemize}
\item Photo-z uncertainties specific to cluster fields present one of the primary systematics in cluster weak-lensing analyses to anchor the absolute cluster mass calibration (Applegate et al. 2014, DESC white paper).  Galaxies along a line of sight that contains a cluster are more likely to be at the cluster redshift than would be expected based upon posterior probability distributions calculated without any reference to position.  Such modulation of the prior should depend upon both cluster redshift and mass.  Some amount of training datasets (targeting galaxies selected by their photo-z's to be behind clusters) is necessary for this work to identify (possibly unexpected) failure modes of mistaking cluster galaxies for background galaxies. FOVs of $\gtrsim 20$~arcmin are ideal for targeting cluster fields at $z_{\rm Cl}\gtrsim0.2$.  Constraints on the wavelength coverage are somewhat reduced for investigating only the cluster redshift failure mode. Accordingly, existing instruments (e.g. DEIMOS) are already well-suited for this work, especially if the large-scale photo-z training surveys are predominantly performed with larger-area, sparser spectrographs.  Training sets should target clusters over a range of redshift, acquiring ~500 background redshifts per cluster to fully characterize the accuracy of $p(z)$ distributions in cluster fields.  In addition, cross-calibration datasets can be a sensitive test for residual cluster galaxy contamination. These are easier to obtain, targeting (bright) red-sequence galaxies.  Spectroscopy of $\sim 100--300$ galaxies in each of $\sim 20--50$ clusters would be useful for this purpose.  
\item The combination of weak lensing with dynamical mass probes (e.g., measurements of galaxy velocities within clusters) can be used to test non-GR theories of gravity, which could provide an alternative explanation for the cosmic acceleration commonly described to dark energy. The same spectroscopy used to characterize photometric redshifts around clusters can be used to measure infall velocities and hence provide constraints on the origin of cosmic acceleration, if the FOVs extend to $\gtrsim 2h^{-1}$ Mpc from the cluster centers.
\item As the most massive collapsed structures in the Universe, the structure and evolution of clusters is sensitive to the properties of dark matter, although is it challenging to separate non-standard-DM signatures from the effects of baryonic physics (Peter et al. 2013). In merging clusters, the effects e.g. of self-interacting dark matter may be most apparent.  LSST (with eROSITA) will identify hundreds to thousands of (major) mergers, but kinematic information is critical to reconstruct the merger histories and geometries. Instruments with $\sim 15-20$~arcmin FOVs and high-multiplexing capabilities are again well matched. {\bf Ask Will to write, and check with Annika.  Should also include bulleticity?}
\end{itemize}

%\subsection{Science Goals}
%
%Describe your science goals here, as in the scientific justification of an NOAO proposal. If you have multiple science goals, you can either describe them all here, or replicate the science, technical, and capabilities sections for each goal. Just create one summary table for the entire program.  
%
%%NOAO guidelines:
%%
%%The scientific justification should explain the overall goals of
%%your program in the context of your field, as well as the importance
%%of your program to astronomy.
%%Writing a good scientific justification is an art.  It takes
%%skill and practice.  And it requires a good scientific idea.
%%This last you must supply but a few general guidelines
%%about proposal writing might still be helpful...
%%
%%\begin{itemize}
%%\item
%%State succinctly and clearly the problem you are trying to solve
%%and the progress that will be made toward doing so if the proposed
%%observations are successful.  If the review panel members have to work hard
%%even to understand what you want to do, they are unlikely to be
%%sympathetic to your proposal.
%%
%%\item
%%Explain clearly why the project is important and how it
%%relates to the broad context and important issues in your field.
%%Many proposals focus too tightly on a specific observational
%%goal (e.g. ``measure the velocity dispersion of this cluster of galaxies'')
%%without explaining why it is important or how it relates to a
%%significant question about the Universe.
%%
%%\item
%%Be specific.  If your observations will ``constrain theoretical
%%models,'' then discuss what will be constrained and why those
%%constraints matter.  Make sure the review panel understands exactly why
%%the observations you propose will make a difference in your field,
%%and exactly how the observations will refine or
%%require changes in the theory.
%%
%%\item
%%Keep it simple.  Try to focus on the central idea of your proposal.
%%Complex arguments are hard to explain and hard for the panel members to follow.
%%Distracting tangential arguments obscure the theme of your proposal.
%%
%%\item
%%Include a figure to help explain what you want to do.  Sample
%%data or model predictions shown in a figure often help clarify
%%complex arguments for the panel members.
%%
%%\item
%%Keep it short.  Never exceed a page for the text of the scientific
%%justification, and never reduce the font size.  It may even help to
%%be a little under a page, and increase the font size a little!
%%Organize your presentation with paragraphs, headings, and bullets
%%so it is easy to read.
%
%
%
%\subsection{Technical Description }
%
%Give a technical description of your program, describing e.g., sample size and properties, justification of spectral or spatial resolution, wavelength, target density, etc.
%
%
%%NOAO guidelines:
%%The review panel looks to this section to find out about the overall
%%strategy of your observational program.  Why do you need the telescopes
%%and instruments you request? How are your targets selected?
%%Why do you need spectroscopy or imaging, and what measurements will
%%you make from the data?  Why is your approach to be preferred over
%%some other approach, what must the minimum sample size be to achieve
%%your scientific goals (and why), and why are your
%%observations likely to be better than previous work in the field?
%
%
%
%\subsection{Needed Capabilities and Estimate of Demand}
%
%Describe the needed capabilities and demand (e.g., estimate of observing time) that flow down from the science and technical considerations. If applicable, describe the time critical nature of the required capabilities (do you need to have the capabilities while LSST is carrying out the survey or can you do the follow up later?) 




{\bf Strong Lens Environment Characterization:}
To achieve stage IV accuracy in constraining cosmology with time delay lenses (q.v. below), the
mass in the immediate environment of the lens and along the line of
sight must be carefully accounted for. LSST will provide photometric
redshifts and stellar mass estimates for all galaxies in the fields of each
cosmographic lens.  
%We can remove the majority of the bias due to
%external convergence by forward modeling
%\cite{WongEtal2011,McCullyEtal2016}. 
However, for some lens fields,
photometric information may not be enough; multi-object spectroscopic
data can provide significant improvements in the accuracy of the models.

To predict the external convergence for an individual lens to few-percent uncertainty, the $\sim 30$ most important galaxies need to be modeled in full, making
them prime targets for spectroscopy \citep{McCullyEtal2016}.  Velocity
dispersion measurements for the lens galaxies allow us to break the lens radial profile degeneracy, a 
key source of systematic uncertainty, while determining the velocity
dispersions of other foreground galaxies will allow more
effective forward modeling, reducing the residual bias in
external convergence estimates by a factor of two
\citep{CollettEtal2013}.  


The technical requirements for this strong lensing program are
summarized below, and can be seen to be less stringent than those listed
in Section~\ref{sec:photoz_design}, suggesting that the same instruments could be used, if fibers can be packed densely enough on the sky:  
$\bullet$ Spectral coverage: as broad as possible, preferably from the $u$ band
through $K$.
$\bullet$ Spectral resolution: can be low, $R$ of a few hundred to a few
thousand (with $R>2000$ necessary for velocity dispersion measurements).
$\bullet$ Field of view: at least 5~arcmin, preferably more.
$\bullet$ Number of fields: At least 100, with $\sim10-100$ targets per field.

Observational costs might be reduced by embedding strong lensing observations in photometric redshift training and/or galaxy evolution surveys.  However, because of the association of strong lensing with massive galaxies in a limited redshift range, requiring the presence of strong lens systems would bias redshift distributions and cause surveys not to be a fair sampling of sample/cosmic variance, and thus may not be practical.

Additional spectroscopy can be useful for characterizing the general effects of weak lensing on  strong lensing measurements; this has similar
requirements as modeling intrinsic alignments, and can find strong
synergy with photo-z training surveys.
% Question from JAN: is this supposed to be in the same fields as the strong lens systems, or can it be done anywhere?  Only in the latter case can it have synergies with training surveys.

%
%There are significant gains to be made by combining these observations
%with other redshift surveys, such as those for photometric redshift
%training or calibration (Chapter~\ref{chp:photoz}). However, timing is
%an issue:  the first LSST lenses will not be available until 2024. The
%30 or so systems studied in the Stage III time delay cosmography program
%will make better targets for  multi-object spectroscopic surveys carried
%out prior to LSST DR1.
% COMMENT FROM JAN: Timing is not something I'm worried about -- most photo-z training spectroscopy will extend past 2024, I'd bet -- but rather the fact that we need fair samples of the universe for photo-z work, which requiring presence of lens systems would prevent.


{\bf Ambiguous blends:}
Approximately 14\% of the objects detected in the LSST survey will be ambiguous blends of two or more galaxies \citep{Dawson}. These  blends pose a challenge and potential source of systematics for photometric redshift algorithms which assume that all objects are single, isolated galaxies.
%By definition these ambiguous blends will be undetectable in ground-based imaging, so it is likely that the best strategy for mitigation will be calibration of the effects.  These objects should have little effect on cosmological measurements from strong lensing, supernovae or LSS, but could have larger impact on weak lensing measurements (including cluster mass calibrations).
The impact can be predicted given knowledge of galaxy redshift distributions, colors, sizes, and clustering -- knowledge that should be provided by the photometric redshift training survey described above; additional spectroscopy with an equivalent instrument can help to test models of the effect.  For example, one could use overlapping space-based imaging, ideally in both field and cluster environments, to identify a sample of ambiguous blends and measure their redshifts (either in parallel to or after the photo-z training survey).  Because the objects used to test algorithms need not be a fair sample of the universe, this work could be done with smaller field-of-view instruments than those needed for photometric redshift training; a number of current (e.g., Keck/DEIMOS) or planned (e.g., TMT/WFMOS) spectrographs could fulfill this need.

