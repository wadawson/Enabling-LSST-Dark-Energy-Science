\subsection{Science Goals}

Observations of Type Ia supernovae (SNe~Ia) led to the discovery that
the Universe's expansion is currently accelerating
\citep{Riess98:Lambda, Perlmutter99}.  SNe~Ia continue to be a mature
and important cosmological tool \citep[e.g.,][]{Suzuki12, Betoule14,
Rest14}.  Further observations of SNe~Ia will be critical to
improved understanding of the nature of dark energy, perhaps the most
puzzling open problem in all of physics.

LSST will detect and observe $\sim$ $10^{6}$ SNe~Ia out to $z \approx
1$.  With these data, we will be able to measure precise distances and
constrain cosmological parameters.  However, dark energy constraints
are not currently limited by statistics, and even $10^{3}$ SNe~Ia are
more than necessary to reach the current systematic floor
\citep{Betoule14, Scolnic14:ps1}.  While LSST will certainly reduce
some systematic uncertainties (such as those related to calibration),
those related to astrophysics (differences in the SN observables, the
unknown nature of dust, etc) can also be addressed with the proper
auxiliary information.

There are two approaches to using large samples of SNe~Ia for
cosmology.  The first, which has been the standard for more than two
decades, is to use a sample of spectroscopically confirmed SNe~Ia.
The second, which has only just begun to be used for cosmology
\citep{Campbell13}, is to use photometrically classified SNe~Ia.  The
former guarantees that the sample is ``pure,'' consisting of only
genuine SNe~Ia, while the latter is more ``complete,'' but at almost
certainly an equal or lower purity level.  The assumption is that SN
cosmology with LSST will primarily use a photometric sample.

It has been shown that knowing the redshift of a SN significantly
improves its classification and will also reduce distance scatter
some.  Nonetheless, we can make a first pass at classification with
only light curve information (with perhaps including a prior using
photometric redshifts).  The expected path to cosmology will likely
require host-galaxy redshifts.  While a subset of $\sim$ 10\% of the
SNe will be ``hostless'' (having a host galaxy fainter than the
detection limit of the reference image), most host galaxies could be
targeted after the SN has faded.  For the hostless SNe, on the other
hand, we must generally obtain a redshift from the SN itself.  If one
worries about having an unbiased sample, obtaining SN spectra of
hostless SNe would be a priority.

Furthermore, spectroscopy of the SNe themselves provide a test of
photometric classification, improved distance precision
\citep{Bailey09, Blondin11, Foley11:vel}, and additional information
about the explosion.  A subset of SN~Ia spectra will be necessary to
perform the most precise cosmology analysis.

\subsection{Technical Description}

\subsection{Needed Capabilities and Estimate of Demand}
