{\it Ryan Foley, XYZ and Jeffrey Newman }


\subsection{Science Goals}

Observations of Type Ia supernovae (SNe~Ia) led to the discovery that
the Universe's expansion is currently accelerating
\citep{Riess98:Lambda, Perlmutter99}.  SNe~Ia continue to be a mature
and important cosmological tool \citep[e.g.,][]{Suzuki12, Betoule14,
Rest14}.  Further observations of SNe~Ia will be critical to
improved understanding of the nature of dark energy, perhaps the most
puzzling open problem in all of physics.

LSST will detect and observe $\sim$ $10^{6}$ SNe~Ia out to $z \approx
1$.  With these data, we will be able to measure precise distances and
constrain cosmological parameters.  However, dark energy constraints
are not currently limited by statistics, and even $10^{3}$ SNe~Ia are
more than necessary to reach the current systematic floor
\citep{Betoule14, Scolnic14:ps1}.  While LSST will certainly reduce
some systematic uncertainties (such as those related to calibration),
those related to astrophysics (differences in the SN observables, the
unknown nature of dust, etc) can also be addressed with the proper
auxiliary information.

There are two approaches to using large samples of SNe~Ia for
cosmology.  The first, which has been the standard for more than two
decades, is to use a sample of spectroscopically confirmed SNe~Ia.
The second, which has only just begun to be used for cosmology
\citep{Campbell13}, is to use photometrically classified SNe~Ia.  The
former guarantees that the sample is ``pure,'' consisting of only
genuine SNe~Ia, while the latter is more ``complete,'' but at almost
certainly an equal or lower purity level.  SN
cosmology with LSST is expected to primarily use photometric samples.

It has been shown that knowing the redshift of a photometric SN significantly
improves its classification and will also reduce distance scatter
some.  Nonetheless, we can make a first pass at classification with
only light curve information (with perhaps including a prior using
photometric redshifts).  The expected path to cosmology will likely
require host-galaxy redshifts.  While a subset of $\sim$ 10\% of the
SNe will be ``hostless'' (having a host galaxy fainter than the
detection limit of the reference image), most host galaxies could be
targeted to obtain spectroscopic redshifts after the SN has faded, which can be done efficiently using multiobject spectrographs.  

For the hostless SNe, on the other
hand, we must generally obtain a redshift from the SN itself.  Hence, if one
desires an unbiased sample of all SNe~Ia, obtaining spectra of
hostless SNe is a necessity.  Furthermore, spectroscopy of SNe themselves provides tests of
photometric classification, improved distance precision
\citep{Bailey09, Blondin11, Foley11:vel}, and additional information
about the explosion physics.  Spectra of a subset of active SNe~Ia will therefore be necessary to
perform the most precise cosmology analysis.

\subsection{Technical Description}

We consider three different applications of spectroscopy for supernova cosmology separately: targeted spectroscopy of SN~Ia hosts after supernovae have faded; targeting active supernovae during the course of other wide-field spectroscopy; and single-object spectroscopy of individual supernovae, both to measure redshifts for hostless supernovae and to improve our understanding of supernova physics.  All of this work will be dependent upon continued support for a transient broker to prioritize SNe~Ia for follow-up observations (see Chapter XYZ);  it will be important to be able to obtain samples of active or historical SNe~Ia on demand whenever spectroscopic resources are available.   

{\bf Wide-field multi-object spectroscopy:}  Spectra of host galaxies are useful for all SNe~Ia.  For supernovae without spectroscopy, they can determine the redshift, but even for those of known redshift, spectra
can provide information about star-formation rates or other quantitites which can
help improve distance measurements \citep[e.g.,][]{Pan14}.  

Most of the host galaxies of the $\sim 10^{6}$ supernovae found by LSST will be faint enough to
require significant exposure times on large telescopes; observing a nonnegligible fraction thus requires multi-object spectrographs.      
%If they each
%required an average of 30~min exposure time, follow-up
%observations would total $\sim$ 137 years performing each in series.
%Instead, we expect to use a wide-field multi-object spectrograph to
%obtain these data.
We expect to discover roughly 100 SNe~Ia~deg$^{-2}$~year$^{-1}$ in the 10--20 LSST
deep-drilling fields; these objects will have the best light curves and be most useful for constraining cosmology.  In principle one could wait until the end of the survey to obtain SN host redshifts (when the density is roughly
1000~deg$^{-2}$), it would be impractical to wait that long.  
%The combination of other
%spectroscopic surveys in these fields, existing galaxy catalogs, and
%live SN observations will reduce this density.  Moreover, we will
%certainly want to perform some science before the end of the survey.

Since the aim of SN host spectroscopy is to measure galaxy redshifts, the instrumental requirements are very similar to those for photometric redshift training spectroscopy, as described in \S \ref{sec:photoz_design}.  Because the supernova host samples are dilute within fields that are $\sim 10$ square degrees each, a wide field of view (preferably $>1$ sq deg) is essential.   Hence, as some LSST deep-drilling fields are at too low declination for effective observations from the North, maximizing host galaxy samples will require a new wide-field multi-object spectrograph with capabilities like those described in \S \ref{sec:photoz_design} in the South.


{\bf High-throughput, Wide-wavelength Optical(/NIR) Spectroscopy on Large Telescopes:}  Spectroscopy of SNe themselves will be critical to the success of the LSST
SN~Ia cosmology program.  Most LSST SNe will be found near the
magnitude limit of the search images, requiring a $\gtrsim$8m
telescope to obtain high-S/N observations.  The source density is low enough that high-throughput, ``single-object'' spectrographs are best suited for this work, unless other high-priority sources could fill the vast majority of available fibers (q.v. below).  

Since SN features are relatively broad, a low-resolution spectrograph
is adequate; it is most important to have broad wavelength coverage.  This both aids in
supernova type identification and
makes comparisons between samples at different redshifts easier.  Ideally, the spectrograph
would cover all optical and NIR wavelengths from $\sim$ 0.3 --
2.5~$\mu$m.  However, a spectrograph covering the full optical range
($\sim$ 0.3 -- 1~$\mu$m) would be adequate.   It will also be important to minimize contamination by second-order light, which can significantly distort SN spectra.  Such spectrographs will be needed on as many telescopes as possible to take maximal advantage of the LSST supernova sample.

%Because the sky density is relatively low, slitmasks or fiber-fed
%spectrographs are not necessarily a good choice for relatively
%small-field large-aperture telescopes (with the possible exception of
%Subaru).  Instead, a high-throughput ``single-object'' spectrograph is
%ideal.

{\bf Real-time Fiber Allocation:}  DES has been successful at allocating fibers to active transients
during AAT observing runs \citep{Yuan15}.  This has yielded spectra of roughly as
many SNe as all other DES spectroscopic programs combined.  It would be valuable if LSST supernova spectroscopy could similarly piggyback on other spectroscopic campaigns on robotically-positioned wide-field multi-object spectrographs.  
% (yet with a
%lower median redshift, typically lower S/N, and at a broader phase
%distribution).  With our above rate, w



\subsection{Needed Capabilities and Estimate of Demand}

{\bf Wide-field multi-object spectroscopy:}  A reasonable strategy for supernova host spectroscopy would be to observe all possible host galaxies in the LSST deep drilling fields 
roughly once a year.  The density would then be roughly
100--300~deg$^{-2}$ (as some galaxies will not yield redshifts after a
single pass).  The maximum density would fill the available fibers on WHT/WEAVE or 4MOST, but would occupy only $\sim 50\%$ of fibers on DESI or $\sim 20\%$ on PFS.    As a result, it would be most effective to combine multiple science programs, with a large fraction of fibers being reserved for SN host galaxies; this is
currently being done successfully for DES using AAT/AAOmega
\citep{Yuan15}.

The majority of LSST supernova hosts will be brighter than $r=24$, and almost all will be at $z<1.6$.  DESI or an equivalent spectrograph should be able to measure redshifts for the great majority of such objects in $\sim 8$ hours of observation time, hence requiring 15--30 nights per year to cover all of the LSST deep drilling fields (the DESI field of view is modestly smaller than LSST's, but shifting field centers for each year's campaign should enable the vast majority of hosts to be targeted over time).  Smaller field-of-view spectrographs are inefficient for this work as they would need to tile the field rather than covering all hosts simultaneously; it would take Subaru/PFS roughly the same amount of time to achieve an identical signal-to-noise on the same set of supernova hosts that DESI would cover, despite being on an 8m telescope instead of a 4m.  

{\bf High-throughput, Wide-wavelength Optical(/NIR) Spectroscopy on Large Telescopes:}  Characterizing hostless supernovae and obtaining sufficient spectroscopy for detailed exploration of supernova physics should require $\sim$ $3\times10^3 -10^4$ objects in total.  The typical exposure time with an efficient spectrograph on a telescope of suitable aperture for a given object should be $\sim$ 30 minutes.  Hence, total exposure times will be $\sim 1500-5000$ hours, or $\sim 300-900$ nights (including weather losses).  Crudely, we expect $\sim 20\%$ of LSST supernovae to be observable with a 4m telescope, $\sim 60\%$ to require an 8m, and $\sim 20\%$ to be at high enough redshift that observations with a GSMT are strongly preferred.  Hence, the total instrumental need corresponds to 60--180 4m nights, 180--540 8m nights, and 60--180 GSMT nights over the course of the 10-year LSST survey.  To enable this, it will be important that every large-aperture telescope possible in the Southern hemisphere should be outfitted with a multi-purpose instrument that can enable this work; existing examples of such spectrpgraphs have been in extremely high demand, and that should only increase in the LSST era.

{\bf Real-time Fiber Allocation:}  We can expect roughly 10--20 active SNe~Ia per square degree.  Hence, it would be beneficial if whenever wide-field, robotically positioned, fiber-fed spectrographs are aimed at LSST fields, they could allocate $\sim$ 15 fibers~deg$^{-2}$ to observations of active SNe, with targets to be identified shortly before the time of observation.  In this case, the overall exposure time, etc. are set by the primary survey which SN spectroscopy is piggy-backing on, and cannot be estimated separately.