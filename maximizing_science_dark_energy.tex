 \documentclass[12pt]{report}
%\documentclass[10pt]{article}
%\addtolength{\hoffset}{-3cm} \setlength{\topmargin}{-2.5cm} \setlength{\textheight}{25cm}
%\setlength{\footskip}{0.75cm} \setlength{\textwidth}{18cm} \setlength{\evensidemargin}{0.in}
%\linespread{0.9} \pagestyle{plain}



% \addtolength{\hoffset}{-0.35cm} \setlength{\topmargin}{-2cm} \setlength{\textheight}{24cm}
%\setlength{\footskip}{0.75cm} \setlength{\textwidth}{17.5cm} \setlength{\evensidemargin}{1.in}
% \documentclass[12pt,preprint]{aastex}
% \addtolength{\hoffset}{-0.5in}
% \setlength{\topmargin}{-0.75in}
% \setlength{\textheight}{9.4in}
%\setlength{\footskip}{0.75cm}
%\setlength{\textwidth}{16.75cm}
%\setlength{\evensidemargin}{1in}
%\setlength{\oddsidemargin}{0.5in}
%\linespread{1.0}
%\pagestyle{plain}
%\usepackage{graphics}
%\usepackage{color}
%\usepackage{times}
%\setcounter{section}{0}



\usepackage{datetime}
\usepackage{fancyhdr}
\usepackage[outermarks]{titlesec}
\usepackage[utf8x]{inputenc}
\usepackage[T1]{fontenc}
\usepackage{amsmath}
\usepackage{amssymb}
\usepackage{epsfig}
\usepackage{graphics}
\usepackage{graphicx}
\usepackage[usenames]{color}
\usepackage{helvet}
\usepackage{times}
\usepackage{natbib}
\usepackage{import}
\usepackage{tabularx}
\usepackage{xspace}
\usepackage{scrextend}
%\usepackage{parskip}
\usepackage[normalem]{ulem}
\usepackage{scrwfile} % Stops "No room for a new \write" complaint
\usepackage{xcolor}
\usepackage{colortbl}
% Big tables summarizing plans:
\usepackage{pdflscape}
\usepackage{afterpage}

\usepackage{pdftexcmds}

% Hyperref - always load as the last package!
\usepackage[linktocpage=false]{hyperref}
\hypersetup{
    colorlinks=true,
    citecolor=blue,
    filecolor=blue,
    linkcolor=blue,
    urlcolor=blue,
}

\usepackage{hypcap}

%\renewcommand{\thesection}{\Alph{section}}
%\setcounter{page}{1}
%\renewcommand{\thepage}{0-\arabic{page}}
\def\be{\begin{equation}}
\def\ee{\end{equation}}
\def\bea{\begin{eqnarray}}
\def\eea{\end{eqnarray}}
\def\bc{\begin{center}}
\def\ec{\end{center}}
\def\lg{\log_{10}}
\def\tcr{\textcolor{red}}
%\begin{document}


\linespread{1.1}
\addtolength{\hoffset}{-0.0cm}
\setlength{\topmargin}{-0.in}
\setlength{\textheight}{8.25in}
\setlength{\footskip}{1.5cm} %{0.75cm}
\setlength{\textwidth}{6.25in}
\setlength{\evensidemargin}{1in}
\setlength{\oddsidemargin}{0in}

\setlength\parindent{0pt}
\setlength{\parskip}{0.3cm}
%  \usepackage{paralist}
%\setlength{\pltopsep}{0pt}
%\setlength{\plpartopsep}{0pt}

\def\tcr{\textcolor{red}}
\def\tcb{\textcolor{blue}}
\def\accent{\it}
\def\vs{\vspace{0.5cm}}
\def\hs{\hspace{0.5cm}}
\urlstyle{rm}

%--------------------------------------------------------------
%--------------------------------------------------------------
\begin{document}

\pagestyle{fancy}
\fancyfoot{}% clear all footer fields
\fancyfoot[R]{\thepage}  % page number in "outer" position of footer line

\fancyhead[L]{ \rm Maximizing Science in the Era of LSST: Dark Energy Science Cases}
\fancyhead[R]{}
\renewcommand{\footrulewidth}{1pt}

{\bf General guidelines for April 18th: we should produce brief "white papers" describing the science in greater detail and 
		quantifying the capabilities needed. Similar to a rough draft of an NOAO observing proposal. 
		It has the same elements of compelling science, technical description and quantitative 
		statement of resource needs. But does not have to be very polished.} 



\chapter{Science Case 1: Multi-Object Spectroscopy for Training and Calibrating Photometric Redshifts}

%A) Photo-z training and calibration (including in cluster regime, spectroscopy of blends identified in space-based imaging)
%		- JN, AvdL, WD, SS, MD
% ======================================================================

%\section{Photo-z}
\label{sec:photoz}

Many LSST probes of dark energy will be greatly strengthened by the addition of spectroscopic redshift measurements for large samples of galaxies.  Spectroscopic samples to train photometric redshift algorithms will improve constraints from almost all LSST extragalactic science; spectroscopy obtained with the same instruments can be important for investigating the effects of intrinsic alignments between galaxies on the observed weak lensing signal or for improving the use of galaxy clusters as a cosmological probe, amongst other applications.  This work will benefit from spectrographs that maximize multiplexing, areal coverage, wavelength range, resolution, and telescope aperture, as described below.

Larger, wider area samples of bright galaxies and QSOs with redshifts, such as those provided by DESI, can be used to calibrate the actual results of photometric redshift algorithms via cross-correlation measurements; the same samples can contribute to cross-correlation studies of intrinsic alignments and LSST large-scale structure.  We will focus on the photometric redshift algorithms below, as they are broadest in impact and their requirements have been studied in the most detail; however, a broad swath of dark energy science would benefit from the same capabilities.

\section{Scientific Justification}

LSST dark energy and extragalactic science will be critically dependent upon photometric redshifts (a.k.a. photo-z's): i.e., estimates of the redshifts of objects based only on flux information obtained through broad filters.  Improvements that yield higher-quality, lower-RMS-error photo-z's will result in smaller random errors on parameters of interest and enable analyses in narrower redshift bins; while systematic errors in photometric redshift estimates, if not constrained, may dominate all other uncertainties when studying cosmological parameters.  Both optimizing and calibrating photo-z's are dependent upon obtaining secure redshift measurements from spectroscopy.

For convenience, we divide the utility of spectroscopy for photo-z's into two parts: 
\begin{itemize}
\item {\it Training}, that is, making algorithms more effective at predicting the actual redshifts of galaxies, reducing random errors.  Essentially, {\bf the goal of training is to minimize all moments of the distribution of differences between photometric redshift and true redshift}, rendering the correspondence as tight as possible, and hence maximizing the three-dimensional information in a sample; and 

\item {\it Calibration}, the determination of the actual bias and scatter in redshifts produced by a given algorithm (for most purposes, this reduces to the problem of determining the actual redshift distribution for samples selected based on some set of properties).  Essentially, {\bf the goal of calibration is to determine with high accuracy the moments of the distribution of true redshifts of the samples used for a given study}.  
\end{itemize}
Different datasets will be needed for each of these purposes.

\subsection{Training}

{\bf Training} methods generally use samples of objects with known z to develop or refine algorithms, and hence to reduce the random errors on individual photometric redshift estimates. The larger and more complete the training set is, the smaller the RMS error in photo-z estimates should be, increasing LSST's constraining power.  For instance, LSST delivers photo-z's with an RMS error of $\sigma_z \sim 0.025(1+z)$ for $i<25.3$ galaxies in simulations with perfect template knowledge and realistic photometric errors, whereas photo-z's in actual samples of similar S/N have delivered $\sigma_z \sim 0.05(1+z)$. 

With a perfect training set of galaxy redshifts down to LSST magnitude limits, we could achieve the system-limited performance; this would improve the Dark Energy Task Force figure of merit from LSST lensing+BAO studies by $\sim25$\%, with much larger impact on a variety of extragalactic science (e.g., the identification of galaxy clusters).  {\bf Better photometric redshift training will improve almost all LSST extragalactic science, and hence addresses a wide variety of decadal science goals.}  To enable this, we need secure spectroscopic redshifts for as wide a range of galaxies as possible down to the magnitude limits of key LSST samples ($i<25.3$ for the weak lensing ``gold sample'').

\subsection{Calibration}

Similarly, secure spectroscopic redshifts are needed for {\bf calibration}; e.g., the determination of both any overall bias in photo-z's  as well as their true scatter.  {\bf Inadequate calibration will lead to systematic errors in the use of photo-z's, affecting almost all extragalactic science with LSST and hence many decadal science priorities}.  It is estimated that the true mean redshift for each sample used for LSST cosmology (e.g., objects selected to be within some bin in photometric redshift) must be known to $\sim 2\times10^{-3}(1+z)$, i.e., 0.2\%.

If extremely high completeness (>99.9\%) is attained in the spectroscopic samples used for training, then LSST calibration requirements would be met directly. However, existing deep redshift samples have failed to obtain secure redshifts for a systematic 20\%-60\% of their targets; it is therefore quite likely that future small-area, deep redshift samples will not solve the calibration problem. 

Instead, we can utilize cross-correlation methods to calibrate photo-z's.  These techniques cross-correlate the positions on the sky of objects with known $z$ with the positions of the galaxies whose redshift distribution we aim to characterize.  We can then exploit the fact that bright galaxies with easily-measured spectroscopic redshifts trace the same underlying dark matter distribution as fainter objects for which spectroscopy is difficult.    By measuring the cross-correlation signal as a function of spectroscopic redshift, one can determine the $z$ distribution of a purely photometric sample with high accuracy.  The key data needed for photo-z calibration via cross-correlations are redshift samples with very large numbers of objects over a wide area of sky, spanning the full redshift range of LSST targets of interest.

%NOAO guidelines:
%
%The scientific justification should explain the overall goals of
%your program in the context of your field, as well as the importance
%of your program to astronomy.
%Writing a good scientific justification is an art.  It takes
%skill and practice.  And it requires a good scientific idea.
%This last you must supply but a few general guidelines
%about proposal writing might still be helpful...
%
%\begin{itemize}
%\item
%State succinctly and clearly the problem you are trying to solve
%and the progress that will be made toward doing so if the proposed
%observations are successful.  If the review panel members have to work hard
%even to understand what you want to do, they are unlikely to be
%sympathetic to your proposal.
%
%\item
%Explain clearly why the project is important and how it
%relates to the broad context and important issues in your field.
%Many proposals focus too tightly on a specific observational
%goal (e.g. ``measure the velocity dispersion of this cluster of galaxies'')
%without explaining why it is important or how it relates to a
%significant question about the Universe.
%
%\item
%Be specific.  If your observations will ``constrain theoretical
%models,'' then discuss what will be constrained and why those
%constraints matter.  Make sure the review panel understands exactly why
%the observations you propose will make a difference in your field,
%and exactly how the observations will refine or
%require changes in the theory.
%
%\item
%Keep it simple.  Try to focus on the central idea of your proposal.
%Complex arguments are hard to explain and hard for the panel members to follow.
%Distracting tangential arguments obscure the theme of your proposal.
%
%\item
%Include a figure to help explain what you want to do.  Sample
%data or model predictions shown in a figure often help clarify
%complex arguments for the panel members.
%
%\item
%Keep it short.  Never exceed a page for the text of the scientific
%justification, and never reduce the font size.  It may even help to
%be a little under a page, and increase the font size a little!
%Organize your presentation with paragraphs, headings, and bullets
%so it is easy to read.


\section{Experimental Design}\label{sec:design}

\subsection{Training}

A previous whitepaper on {\it Spectroscopic Needs for Imaging Dark Energy Experiments} (Newman et al. 2015) has explored in detail the minimum characteristics a photometric redshift training set should have.  We summarize those conclusions here.  In short, we require:

\begin{itemize}

\item {\bf Spectroscopy of at least 30,000 objects down to the survey magnitude limits}, in order to characterize both the core of the photo-z/spectroscopic-z relationship and the nature of outliers (cf. Ma et al. 2006, Bernstein \& Huterer 2009, and Hearin et al. 2010); given LSST depth, this will require large exposure times on large telescopes.

\item {\bf High multiplexing}, as obtaining large numbers of spectra down to faint magnitudes will be infeasible otherwise.

\item {\bf Coverage of as broad a wavelength range as possible}, in order to cover multiple spectral features, which is necessary for getting highly-secure redshifts (required for photometric redshift training and calibration).  At minimum, spectra should cover from $\sim 4000$ to $\sim 10,000$ Angstroms, but coverage to the atmospheric cutoff in the blue and to $\sim 1.5\mu$m in the red would be advantageous.

\item {\bf Moderately high resolution ($R>\sim 4000$) at the red end}, critical as it enables secure redshifts to be obtained from the [OII] 3727 Angstrom doublet alone.  Because night sky emission lines are intrinsically narrow, high resolution also enables sensitive spectroscopy with low backgrounds over the $\sim 90\%$ of the spectrum that lies between the skylines (assuming that scattered light is well controlled; cf. Newman et al. 2013).

\item {\bf Field diameters $>\sim20$ arcmin}, needed to span multiple correlation lengths to enable accurate clustering measurements.  This is key for enabling the training samples to be well-understood, providing synergistic galaxy evolution science, and enabling the spectra obtained to be utilized with cross-correlation techniques; it also reduces sample/cosmic variance in the training set.  A typical correlation length of $r_0 \sim 5 h^{-1}$ Mpc comoving corresponds to $\sim 7.5$ arcmin at z=1 and 13 arcmin at z=0.5; hence, larger fields would be even better, particularly at low redshifts.

\item And finally, {\bf coverage of as many fields as possible ($\sim 15$ minimum)}, in order to minimize the impact of sample/ cosmic variance.  Cunha et al. (2012) estimated that 40-150 $\sim$ 0.1 deg$^2$ fields would be needed for DES for sample variance not to impact errors; however, we can do somewhat better by taking advantage of the fact that the variance itself is directly observable via the field-to-field variation in redshift distributions.  Nevertheless, with fewer than $\sim 15$ independent (i.e., widely separated) fields, it would be difficult to measure dispersions in counts amongst the fields directly, as measurements of standard deviations are broad and highly skewed for small N. We thus require at least this many fields.
\end{itemize}

In principle, a number of spectrographs currently in existence or being planned could meet these goals (i.e., they have sufficient wavelength coverage and spectral resolution to obtain secure redshifts over a wide range of galaxy properties).  However, the time required to conduct the needed survey will depend on the instrumental and telescope characteristics.  If sky coverage is low, the limiting factor will be how many tilings of the sky will be needed to cover $>15$ fields that are 20 arcmin in diameter (we will refer to this situation as ``field-of-view-limited'' or ``FOV-limited'').  If multiplexing is low, the limiting factor will be how many tilings are needed to reach 30,000 spectra (``multiplex-limited'').  Finally, if both of these factors are high, such that each field need only be observed once, the limiting factor will be how much exposure time it takes the telescope to achieve the desired depth (``Fields-limited'').  

The actual time required for a survey in each of these scenarios will be:

{\it FOV-limited}:  $t_{observe}$ =(280 nights) $\times$ ($N_{fields}$ / 15) $\times$ (314 arcmin$^2$ / area per pointing )$\times$( Equivalent Keck/DEIMOS exposure time / 100 hours) $\times$ (0.67 / observing efficiency) $\times$ (76 m$^2$  / telescope effective collecting area) $\times$ (0.3 / [telescope + instrument throughput]),

{\it Multiplex-limited}:  $t_{observe}$ =(280 nights) $\times$ ($N_{objects}$ / 3$\times10^4$) $\times$ (2000 / Number of simultaneous spectra)$\times$(Equivalent Keck/DEIMOS exposure time / 100 hours) $\times$ (0.67 / observing efficiency) $\times$ (76 m$^2$  / telescope effective collecting area) $\times$ (0.3 / [telescope + instrument throughput]), or

{\it Fields-limited}:  $t_{observe}$ =(280 nights) $\times$ ($N_{fields}$ / 15) $\times$ (Equivalent Keck/DEIMOS exposure time / 100 hours) $\times$ (0.67 / observing efficiency) $\times$ (76 m$^2$  / telescope effective collecting area) $\times$ (0.3 / [telescope + instrument throughput]), 

where $t_{observe}$ is the total clock time required for observations (including overheads and weather, which both reduce observing efficiency); $N_{fields}$ is the total number of fields to be observed; and $N_{objects}$ is the total number of objects to be observed.   For any given telescope/instrument combination, the actual exposure time required will be the greatest out of the fields-limited, FOV-limited, and multiplex-limited values.  100 hours' Keck/DEIMOS exposure time would be sufficient to achieve the same signal-to-noise for $i=25.3$ objects that DEEP2 obtains at $i \sim 22.5$; at that magnitude, DEEP2 obtained secure redshifts for $\sim 75\%$ of targets.  A spectrograph with broader wavelength range or higher spectral resolution would be expected to do at least as well as this at equivalent signal-to-noise.

Newman et al. 2015 tabulates the properties of a variety of current and upcoming spectrographs and estimates the total time they would require to complete such a survey.  It would take more than 10 years on Keck with DEIMOS, 5 years with DESI on the Mayall telescope, roughly 1.8 years with TMT/WFOS, just over 1 year with PFS on Subaru, or as little as 4-5 months on GMT or E-ELT, depending on instrument availability and characteristics (for instance, the specifications for the proposed MANIFEST fiber feed for GMT; if it is not available, the survey duration would need to be substantially longer than this).  If less telescope time is available, it will be necessary to either allow spectroscopic redshift failure rates to increase or to reduce the depth of the sample; it is likely that the former would have smaller impact on photometric redshift training, but that question requires further investigation.

By design, this training sample would be sufficient to meet LSST calibration requirements if spectroscopic redshift failure rates of order $\sim 0.1\%$ could be achieved.  However, based on past experience that is far from a sure thing, so we require other spectroscopy for calibrating photometric redshifts.  

We note that the results of this survey -- a set of galaxies spanning the full range of galaxy properties down to the LSST magnitude limit with a maximal amount of spectroscopic information -- will enable a wide variety of galaxy evolution science going well beyond just the training of photometric redshifts.  A number of applications for such a sample are discussed in Chapter XYZ.

\subsection{Calibration}

As described in Newman et al. 2015,  cross-correlation calibration of photometric redshifts for LSST should require spectroscopy of a minimum of $\sim 5 \times 10^4$ objects over multiple independent $>100$ square degree fields, with coverage of the full redshift range of the objects whose photometric redshifts will be calibrated (for LSST dark energy samples, this is essentially $0<z<3$).  Such a sample should be available via overlap between LSST and planned baryon acoustic oscillation experiments.  Those projects are optimized by targeting the brightest galaxies at a given redshift over the broadest redshift range and total sky area possible; this matches the needs for cross-correlation measurements well.  For photometric redshift calibration, all redshifts used must be highly secure ($>$99\% probability of being correct); however, here incompleteness is not an issue.

The DESI baryon acoustic oscillation experiment should obtain redshifts of $>\sim$30 million galaxies and QSOs over the redshift range $0 < z < 4$ over more than 14,000 square degrees of sky (INSERT REFERENCE HERE).  It is expected to have $>4000$ square degrees of overlap with LSST.   Absorption-line systems detected in DESI QSO spectra can provide even more redshifts at low z (Menard et al. 2013).  We anticipate that DESI will provide several million redshifts within the footprint of LSST.  As can be seen in Figure XYZ, DESI will allow detailed analyses of photometric redshift calibration, rendering calibration errors negligible and providing multiple cross-checks for systematics (e.g., by comparing redshift distributions re-constructed using tracers of different clustering properties, or restricting to only the most extremely secure redshifts).  

However, one potential concern is that the DESI survey will cover only the northern portion of the LSST sky, which LSST will tend to observe at relatively high airmass; as a result, photometric redshift performance may not be the same there as elsewhere in the LSST footprint, which could yield a miscalibration when applied to LSST as a whole.  This risk can be mitigated by obtaining DESI-like spectroscopy in the south.  There are plans to conduct a DESI-like survey with the 4MOST telescope that would fulfill this need well.  If this does not happen, it would be extremely valuable to have access to a DESI-like spectrograph (with wide field of view, $\sim 5000\times$ multiplexing, full optical coverage, sufficient resolution to split [OII], and a $\sim 4m$ diameter telescope aperture) in the Southern hemisphere.




%
%NOAO guidelines: 
%The review panel looks to this section to find out about the overall
%strategy of your observational program.  Why do you need the telescopes
%and instruments you request? How are your targets selected?
%Why do you need spectroscopy or imaging, and what measurements will
%you make from the data?  Why is your approach to be preferred over
%some other approach, what must the minimum sample size be to achieve
%your scientific goals (and why), and why are your
%observations likely to be better than previous work in the field?



\section{Summary of Capabilities Required}

{\bf Training: Highly-multiplexed Multi-object Spectroscopy on Large Telescopes}: For training photometric redshifts, it is desirable to have an optical or optical + near-infrared multi-object spectrograph which maximizes all of spectroscopic coverage, multiplexing, field-of-view, and telescope aperture, with modestly high resolution in the red/near-IR.  Minimum requirements to maximize observation efficiency are:

$\bullet$ Spectral coverage: 4000-10,000 Angstroms (3700 Angstroms - 1.5 microns preferable);

$\bullet$ Resolution: R$\sim 4000$ in the red (R$\sim 6000$ preferable); and

$\bullet$ Field of view: $>\sim$20 arcmin diameter ($>$1 degree diameter preferable).

Multiplexing and telescope aperture should then be made as large as practical.  In particular, for most instruments with sub-degree fields of view, the product of multiplexing and telescope area will determine the total observing time needed to assemble a photometric training sample, so this product should be maximized.  

{\bf Calibration: Wide-field, Massively-Multiplexed Spectroscopy}: If a DESI-like sample is not available over the southern portion of the LSST footprint, it will be necessary to either move DESI to the South or develop a comparable instrument at the Blanco or an equivalent telescope (with wide field of view, $\sim 2000--5000\times$ multiplexing, coverage of $\sim 4000--10,000$ Angstroms, and sufficient resolution to split the [OII] doublet, $R>\sim 4000$).  In any event, consideration will need to be given to how external spectroscopic resources should be integrated with LSST data; see Chapter \ref{sec:datacomp} for details.


\section{Other Science Enabled by These Capabilities}

%\subsection{Photometric redshift training and calibration}\label{subsec:training}
%{\em A) Photo-z training and calibration (including in cluster regime,
%spectroscopy of blends identified in space-based imaging).}

% For each science area, we should:
%
% 1) Describe the science need in a few sentences at most
%
% 2) Describe the needed capability (or capabilities) in enough detail
% that someone could determine whether they overlap with other needs.
% (E.g.:  “medium-resolution (R~4000-6000) spectroscopy covering the
% full optical window for i<25.3 objects with multiplexing of ~1000 over
% ~10 arcminute diameter fields” could work, or for a differentcase
% “optical medium-resolution spectroscopy with a highly-multiplexed
% spectrograph with a many-degree FoV on a 4m telescope” would also give
% enough information to allow identification of common needs).

% ----------------------------------------------------------------------

\subsection{Ambiguous Blending}
Approximately 14\% of the objects detected in the LSST survey will be ambiguous
blends of two or more galaxies (Dawson et al. 2016); that is, galaxies with
separations $\lesssim$ their FWHM, such that they cannot reliably be identified as blended and are detected as a single object. These ambiguous blends pose a challenge and
potential source of systematics for photometric redshift algorithms operating
under the assumption of a single isolated galaxy (currently all photo-z
algorithms). By definition these ambiguous blends will go undetected, so it is likely that the best strategy for mitigation will be calibration of the effects.  One option  is to use overlapping space-based imaging in both the field and cluster
environments to identify a sample of ambiguous blends and observe these galaxies
with a high-multiplexing, medium resolution spectrograph with large enough slits or fiber heads to encompass both components of a blend.
%
%{\em Note: ultimately, would need some estimate of sample size.  I'll admit that I'm not convinced a spectroscopic campaign focused on blends would be very efficient nor provide a better way of characterizing the effect than predicting it from clustering measurements + redshift distributions + space-based imaging (since the key question is what fractions of blends are same-redshift vs. different-z). --JAN}
% Since resolution in the redshift
%dimension is all that is needed for such a study, a simple slit-based
%spectrograph (with the slit properly positioned), and potentially a fiber-based
%(assuming the fiber encompasses both/all galaxies), will be sufficient and with
%multiplexing will be more efficient than an AO IFU.



%B) Weak lensing: Intrinsic alignment studies, exquisite-seeing data for morphology templates, etc.
% RM, WD
\section{Weak lensing}
\label{sec:wl}

{\it B) Weak lensing: Intrinsic alignment studies, exquisite-seeing data
for morphology templates, etc.}

\subsection{Galaxy morphologies}

After some discussion, we conclude that our report doesn't have to say anything about calibrations
of galaxy morphology and ellipticity distributions, because we will get what we need either from
LSST data itself or from HST.  (For longer discussion of why this is the case, see the version of
this document in commit b29e99f64e6633218f3316092b6d6fdb38f6d209.)

\subsection{Intrinsic alignments}

Intrinsic alignments (IA) of galaxy shapes with the cosmic web are a contaminant to weak lensing
measurements, since WL measurements assume that all coherent alignments are due to lensing. IA have
been robustly detected out to hundred Mpc scales, making them a serious theoretical uncertainty for
WL cosmology contaminating both density-shape and shape-shape correlation functions.  Several
methods have been developed for mitigating this systematic, including nulling (which loses a lot of
cosmological information and leads to stringent constraints on photo-$z$ errors), forward modeling
(which uses galaxy clustering, galaxy-shear, and shear-shear correlations measurements, and involves
marginalizing over a model for how the alignments enter those measurements), and self-calibration
(which does not require an {\em a priori} alignments model, but has certain assumptions that have
not yet been validated).

Current data, primarily from the SDSS, provide us a template for how intrinsic alignments scale with
galaxy type, luminosity, and redshift for $z\lesssim 0.5$.  However, we need this information for
galaxies at higher redshifts, and we need better constraints on the alignments of blue galaxies, in
order to place reasonable priors when carrying out the WL analysis with LSST. Generally, this
requires both good imaging and redshift information, in order to better localize the galaxy pairs in
3D and measure galaxy shapes.  Thus, the LSST data itself will ultimately be quite informative about
IA despite the size of the photo-$z$ errors, but there is still value in strong external priors on
the IA model because IA are degenerate with other systematics (photo-$z$ errors and certain shear
systematics).

In principle a direct measurement of IA requires a spectroscopic dataset that covers decent-sized
contiguous areas (so as to constrain alignments to tens of Mpc) but also has a decent sampling rate
of a {\em representative galaxy sample} within these fields.  This seems somewhat orthogonal to a
``many small fields to beat down cosmic variance'' approach that one might want for spectroscopic
observations to constrain photo-$z$ errors (see the ``training'' dataset described in
Sec.~\ref{subsec:training}), but perhaps a compromise could be reached.  Another question is whether
we could tolerate a more sparse and/or unrepresentative sampling of the density field over a wider
area, and constrain IA using cross-correlations with the spectroscopic galaxies instead of the
auto-correlations of them (using a method like that from Blazek et al 2012 or Chisari et al 2014).
The dataset needed for this could well look something like what is described in
Sec.~\ref{subsec:training} as the dataset for enabling the photo-$z$ cross-correlation method. We
will explore what compromises can be made while still supporting the LSST dark energy science at the
needed level in the coming weeks.


%C) Strong lensing: Monitoring, spectroscopy, positions (incl. IFU spectroscopy, monitoring of lens solution for supernovae, high resolution imaging follow-up with ELTs, spectroscopy to enable combined SL/WL analysis of clusters)
% PM, AB, WD, TT?, EL with input from the DESC SL WG

% ======================================================================

\subsection{Galaxy Clusters}
\label{sec:clusters}
{\it Anja von der Linden, Will Dawson, Elise Jennings and others}

The abundance of massive clusters of galaxies is sensitive to cosmological parameters as it is fundamentally detemined by the abundance of massive dark matter halos.  Small changes in the rate of growth of dark matter overdensities (which, if general relativity is correct, is a simple function of basic cosmological parameters) will yield exponential changes in the mass function on the high-mass tail.  As a result, measurements of the cluster mass function will be a statistically very powerful technique in the LSST era, with samples of tens to hundreds of thousands of clusters over a wide range in redshift.  However, harnessing this statistical power requires a detailed understanding of all the systematic uncertainties involved.  

The by far dominant sources of systematic uncertainties relate to understanding of the relation between the observables captured by different measurement methods -- cluster richness for LSST-based cluster finders, but also the Sunyaev-Zeldovich signal used in mm-based cluster samples available in the LSST era (e.g. AdvACT, SPT-3G, CMB-S4), as well as X-ray flux from X-ray selected cluster samples (e.g., eROSITA) -- and the true virial masses of the clusters' host dark matter halos.  LSST is uniquely positioned to address the calibration of mass-observable relations due to its weak-lensing and photo-$z$ capabilities, which can be used for an accurate and precise calibration of cluster masses (DESC 2012, 2015).  The needs for cluster cosmology in terms of additional ground-based OIR data are therefore intricately linked with those of weak lensing and photometric redshifts.  The largest OIR effort we foresee that would be primarily driven by cluster cosmology is training and tests of photometric redshift algorithms in extreme environments.  This work would require instrumentation equivalent to that required for photometric redshift training and calibration (cf. \S \ref{sec:design}).


% ----------------------------------------------------------------------

\subsubsection{Photometric Redshifts in Cluster Fields}

Photo-z's in cluster fields need to perform two functions: measure the cluster redshift, and cleanly separate background galaxies to be used in the weak-lensing measurements from cluster and foreground galaxies.  The latter presents the largest empirical uncertainty on weak-lensing cluster mass estimates, and is thus the primary driver for ancillary datasets for cluster count cosmology with LSST. 

\begin{itemize}
\item {\it Photo-z calibration in cluster fields:} Since the shear induced on a background galaxy is a function of the galaxy redshift, the photo-z's of galaxies in cluster fields need to be accurately known.  It has been shown that while the use of simple one-point estimators leads to significantly biased results, using the full photo-z redshift probability distribution $p(z)$ yields nearly unbiased mass measurements (Applegate et al. 2014).  In addition to general requirements on $p(z)$ distributions (Sect. ?), the distributions need to accurately reflect the probability of a given galaxy to be in the cluster.  This can be achieved e.g. by adding peaks at the cluster redshift into the $p(z)$ priors; the normalization of such peaks then needs to be calibrated, ideally on spectroscopic training sets. {\it AvdL comment: need to figure out prospects for cross-correlation methods for clusters}
\end{itemize}


In terms of determining cluster redshifts, techniques based on red-sequence galaxies have been shown to perform exceptionally well at least to at least $z\le0.8$ (Rykoff et al. 2016), and are expected to yield reliable redshifts to $z\le1.2$ with LSST. 
\begin{itemize}
\item {\it Cluster redshifts at $z\ge1.1$:} Red-sequence-based methods recover accurate cluster redshifts if the red sequence is (1) already well-established, and (2) falls within two bands used in the survey. At higher redshifts, neither of these is necessarily the case.  Yet, the high-redshift clusters are particularly valuable for probing dark energy.  NIR imaging, as well as follow-up spectroscopy, can significantly improve the cluster redshift estimates in this regime.  Gathering such datasets can be done already before completion of LSST by targeting SZ-selected cluster samples.  {\it AvdL comment: Euclid and WFIRST can probably do a lot of this?})
\end{itemize}


\subsubsection{Cluster density profiles}

Clusters with multiple mass probes covering a range in radial range can be used to reconstruct the cluster density profiles, which serves as a validation of the density profiles assumed in the weak-lensing mass measurements, and could be used to test the properties of dark matter, if the astrophysical processes at cluster cores can also be understood (Peter et al. 2013).  In addition to the LSST weak-lensing measurements, which best constrain cluster masses on $\sim1$~Mpc scales, such datasets should include BCG velocity dispersion profiles, spectroscopic redshifts of any strongly lensed multiple image systems, and relatively deep X-ray imaging. In terms of OIR capabilities, this translates to deep long-slit BCG spectra, as well as dedicated efforts to follow-up cluster strong lensing features with optical/NIR multi-object spectroscopy.  {\it AvdL comment: The best candidate clusters will probably all well known before LSST 3-year?}

\subsubsection{Modified gravity}

The combination of weak lensing and dynamical mass probes (e.g. cluster galaxy velocities) can be used to test modified gravity theories.  {\it AvdL comment: there are many version of this: redshift-space distortions, stacked measurements, etc.  How many redshifts per cluster do we actually need?}

{\it Text from Elise:} 
Recently, there has been growing interest in models that also modify the way in which the lensing potential depends on matter density perturbations, such as Nonlocal gravity, Galileon gravity, massive gravity, Kmouflage gravity, and several other subclasses of Horndeski's general theory.
These modifications to the lensing signal give rise to a broader range of ways to test gravity.
Models that change the lensing signal can also have a strong impact on the power spectrum of cosmic shear and the lensing of cosmic microwave background (CMB) photons.
Barreria et al 2015 found practically unmodified lensing masses in the Galileon model have interesting implications for tests of gravity that are designed to detect differences between lensing and dynamical mass estimates. Although the Galileon model modifies the lensing potential, this does not translate into modified lensing masses because of the screening. However, outside $R_{200}$ (at distances [2 --20] Mpc/h from the cluster center), the force profile of clusters in the Galileon model can be 10 -- 40\% higher than in $\Lambda$CDM, which can affect the velocity distribution of infalling galaxies, and hence, the dynamical mass estimates. 
The existence of screening mechanisms in modified gravity models motivates research into the lensing effects associated with cosmic voids. There, the density is low, and as a result, the fifth force effects are manifested more prominently. In the context of modified gravity, voids found in simulations are on average emptier in e.g. $f(R)$ gravity than in $\Lambda$CDM. This happens because the enhanced gravity facilitates the pile up of matter in the surrounding walls and filaments, leaving less mass inside the void. This translates into a stronger signature in the lensing signal from voids.

For a given cosmology with a smooth dark energy component, a measurement of the expansion history gives a prediction for the growth rate of structure. Independent measurements of the growth rate can be obtained by measuring the clustering of galaxies or clusters in redshift space, where peculiar velocities distort the clustering signal along the line of sight. By testing the consistency between the measured growth rate and the prediction from the expansion history, it is possible to constrain models of modified gravity and to distinguish them from a smooth dark energy component.
Many studies (Zu et al. 2014, Jennings et al. 2012, Wyman et al. 2013, etc) find that redshift space distortions in modified gravity models have an impact on the clustering of dark matter on large and small scales to a level which may be distinguished from general relativity. Current redshift space distortion models, which are valid on quasi linear scales, are accurate enough to extract the non linear matter $P(k)$ in real space from the measured 2D redshift space power spectrum on large scales, allowing us to constrain $f(R)$ modified gravity. The large difference between the predicted velocity P(k) in the $f(R)$ and GR cosmology make this a very promising observable with which to test GR provided that this can be accurately extracted from the redshift space power spectrum. The impact of modified gravity on the galaxy
infall motion around massive clusters (Zu et al 2014),  in combination with weak lensing, also offer
a powerful non–parametric cosmological test of modified gravity. The main systematic uncertainty here arises from the imperfect understanding of the impact of galaxy formation physics on galaxy infall kinematics and to complement
WL as a cosmological test of gravity, this modelling of
galaxy clusters requires overlap with a large galaxy redshift survey,
such as BOSS, eBOSS, the Subaru Prime Focus
Spectrograph, DESI or WFIRST. 

---------------------------------
%
%{\it Note from Anja:  Will suggested that strong lensing and/or galaxy velocities can help improve the cluster mass calibration for cluster count cosmology over the default plan of using weak-lensing.  For cluster counts, I remain skeptical on both accounts. Strong lensing is awesome and gives us very precise mass constraints in cluster cores ($\sim 100$~kpc), but requires significant extrapolation to measure masses on the scales of $R_{500} - R_{200}$ ($\sim 1$~Mpc).  Moreover, there is a selection bias towards clusters that are over-concentrated or aligned along the line-of-sight.  While there is simulation work to account for such biases (e.g. Meneghetti et al.), the small scales that strong lensing probes require a thorough understanding of the gas physics at cluster cores, too, which still seems a daunting task.  I DO think strong lensing has a place in precise measurements of cluster density profiles in selected clusters, which is (1) informative for the weak-lensing work, and (2) can be used to probe the processes at cluster cores (both dark matter and cluster astrophysics, see work by Annika Peter).  Regarding the caustic method, I have only seen one blind test done on simulations (Old et al. 2014, 2015), in which it did not perform particularly well.  Those works showed that in general, velocity-based methods suffer even larger biases and scatters than weak-lensing masses.  I'm willing to believe that things look better if one has several hundreds of velocities per cluster, but that would be a significant investment of telescope time just to be comparable to the weak-lensing constraints.  I actually think that modified gravity is a much better selling point for such datasets.\\
%I've modified the point on cluster redshifts to focus it on high-redshift clusters.  At $z<0.8$, redMaPPer/redMaGiC already perform extraordinarily well, and any overlap with spectroscopic samples is probably enough to further confirm that to z~1.2.\\
%I'm not sure whether Will meant cosmology with the most-massive-cluster test (Mortonson et al. 2013)?  I think this outside the LSST realm, because the most massive high-redshift clusters will first be found by the SZ surveys.  Also, this test requires multiple mass measurement, and I think deep weak-lensing data for such candidates will be gathered before LSST 10-year depth is complete.}


%% ----------------------------------------------------------------------
%
%\subsection{Cluster Mass Function: Spectroscopic Surveying in Cluster Fields}
%\label{sec:sl:clusters}
%{\it Will Dawson and others}
%
%{\it Note from Will: While I have done a fair amount of cluster weak lensing and
%spectroscopy work, this has been focused on dark matter science, I am not
%actually an expert in cluster mass function cosmological constraints. That said
%I have drafted some text that I hope will describe the methods and needed
%capability (or capabilities) in enough detail that someone could determine
%whether they overlap with other needs. I am also includeing all spectroscopic
%cluster studies and those limited to strong lensing; we can move the text later
%if we like.}
%
%{\it Basic Premis}: Cosmology is sensitive to the abundance of rare massive
%clusters observed out to and beyond $z\sim 1$. For example, Takada \& Spergel
%(2014) have shown that a characterization of the most massive clusters can
%improve the constraining power of future cosmic shear surveys (e.g., LSST, and
%WFIRST) by approximately a factor of two, provided the cluster masses and
%redshifts are accurately constrained.
%
%Spectroscopy of cluster galaxies can improve LSST cluster mass function
%constraints in a number of ways:
%\begin{itemize}
%  \item {\it Redshifts of multiply imaged background galaxies}: Combined strong
%  and weak lensing measurements of clusters can improve both the cluster mass
%  and concentration constraints by approximately a factor of two (Umetsu et
%  al.~2010). So it is possible to improve the LSST cluster weak lensing
%  constraints by combining strong lensing measurement of the strongly lensed
%  background galaxies (preseumably identified from space-based observations,
%  such as HST or WFIRST, but potentially from LSST alone). However, redshifts of
%  the multiply lensed galaxies are needed in order to properly constrain the
%  cluster mass distribution. There has been some work recently by Keren Sharon's
%  group on the number of multiply lensed galaxy redshifts that are necessary for
%  a given lens model accuracy. Most of the multiply lensed galaxies are faint $R
%  \gtrsim 23$ and require large diameter telescopes.
%  \item {\it Accurate cluster redshifts}: For cosmological constraints it is
%  necessary to know both the mass and redshift of the clusters. While it is
%  possible to obtain better than average photometric redshifts of clusters due
%  to their being hundreds to thousands of galaxies to beat down the Poission
%  photo-z noise and the tight red sequence color relation, constaints can be
%  improved (how much) with accurate redshifts from spectroscopy. It is possible
%  to accomplish this with one to a few redshifts of the brightest cluster
%  galaxies. The instrument requirements are pretty relaxed for this science
%  since on only needs a few redshifts per pointing of rather bright galaxies.
%  \item {\it Cluster galaxy velocities as a probe of the cluster potential}: The
%  spectroscopic caustic method (Kaiser 1987; Regos \& Geller 1989) is the only
%  other successful stand alone method, in addition to weak lenising, of probing
%  the mass density profile to comparably large cluster radii. This is important
%  since joint CMB+SZE constraints indicate that SZE-derived {\em Planck} masses
%  underestimate the true $M_{500}$ masses by $b=1-\langle
%  M_{Planck}/M_\mathrm{true}\rangle\sim 43\%$ $\sim43\%$ --- far from the $\sim
%  20\%$ expected due to deviations from the influence of feedback, non-thermal
%  pressure, and cluster shapes (Battaglia+ 2012). Of even greater concern is
%  that according to WL studies (Sereno \& Ettori 2015), the bias appears to vary
%  with mass. SZE mass estimates are calibrated in part based on X-ray
%  temperatures with spectroscopy that is only reliable out to $r_{2500}\sim
%  0.25r_\mathrm{vir}$. Recent WL analyses of the {\em Planck} sample, that probe
%  the mass distribution to larger radii, yield $1-\langle
%  M_{Planck}/M_\mathrm{WL}\rangle\sim 30\%$ at the $\sim 10\%$ uncertainty level
%  (von der Linden+ 2014; Hoekstra+ 2015), somewhat relaxing the {\em Planck}
%  tension. Hinting at the importance of having probes of the cluster potential
%  out to large radii ($\sim$ the virial radius). To realize this science would
%  require more than a hundred spectroscopic redshifts per cluster, with a
%  moderate resolution spectrograph. A highly multiplexed moderate resolution
%  spectrograph, that can simulatinously target objects within $\lesssim
%  5$\,arcsecons of one another (due to the dense cluster environment). I (wd) am
%  not sure how many clusters would need to be surveyed to make a significant
%  improvement to planned LSST cluster cosmology.
%  \item {\it Redshift space distortions}: This is a probe of modified gravity,
%  and similar to the spectroscopic caustic method requires hundreds of
%  spectroscpic redshifts of cluster galaxies. The spectroscopic demands are the
%  same as the caustic method as well. I (wd) am not sure how many clusters we
%  would need to make this measurement for.
%\end{itemize}
%
%\ldots
%
%% ======================================================================


\chapter{Science Case 2: AO Imaging and/or IFU Spectroscopy for Strong Lensing Cosmography}

% ======================================================================

\section{Strong Lensing}
\label{sec:sl}

% {\em C) Strong lensing: Monitoring, spectroscopy, positions (incl. IFU
% spectroscopy, monitoring of lens solution for supernovae, high
% resolution imaging follow-up with ELTs, spectroscopy to enable combined
% SL/WL analysis of clusters).}

% For each science area, we should:
%
% 1) Describe the science need in a few sentences at most
%
% 2) Describe the needed capability (or capabilities) in enough detail
% that someone could determine whether they overlap with other needs.
% (E.g.:  “medium-resolution (R~4000-6000) spectroscopy covering the
% full optical window for i<25.3 objects with multiplexing of ~1000 over
% ~10 arcminute diameter fields” could work, or for a differentcase
% “optical medium-resolution spectroscopy with a highly-multiplexed
% spectrograph with a many-degree FoV on a 4m telescope” would also give
% enough information to allow identification of common needs).

% ----------------------------------------------------------------------

\subsection{Time Delay and Compound Lens Cosmography: High Precision Galaxy Mass Models from High Resolution Imaging and Spectroscopy}
{\it Phil Marshall, Tommaso Treu and others}

The primary route to cosmology from strong lensing is time delays in
galaxy-scale lensed quasars and supernovae. Galaxy scale compound lenses
(i.e. systems with two sources at different redshifts) have also been
suggested. We expect to be able to compile samples of several hundred lensed AGN
and lensed SN systems with accurately measured LSST time delays (CITE
Liao et al 2015), and dozens of compound lens systems (CITE Collett et al).

To be useful as probes of cosmological distances, galaxy scale lenses
need very well constrained mass models. These constraints will come from
two sources:
\begin{enumerate}
\item High resolution imaging of the Einstein rings due to the source AGN
or SN host galaxy. Targeted snapshot (i.e.\ 200--2000 second exposure time) imaging
observations in the near infra-red with either JWST or GSMTs
can provide the Einstein ring constraints needed to turn
each of these systems into a  5\% precision distance CITE{MengEtal2015}.
\item Spatially resolved spectroscopy of the lens
galaxy, enabling measurement of the stellar velocity dispersion field to
break the degeneracy between the predicted time delays and the lens mass
density profile.
\end{itemize}
These can be obtained simultaneously with integral field units: we will
need capabilities such as those currently available on Keck, and that
will be available on TMT.

However, before this we will need optical spectroscopy to get the redshifts
of the deflector and source galaxies. This can be done at relatively low
resolution, but the wider the wavelength range the better: we will need
3500--10000$\A$ in the optical, and maybe a triplespec covering the JHK
bands in the IR. Some redshifts may come from overlapping
spectroscopic survey observations, with the systems appearing as
composite absorption and emission line galaxy spectral components. Some
of the targeted spectroscopy could be done at 10-m class facilities.

After the redshift has been determined, AO-assisted integral field
spectroscopy would be best for providing the lens mass model
constraints (from both the Einstein rings and the spatially resolved
spectroscopy). We will need something like a 4"x4" field of view, with
spectral resolution R~4000--5000 over a wavelength range of 1.0--2.2 microns
(to cover CaT at rest frame  8800--9000A, or CO at 1.5-1.6$\mu$m). IRIS
on TMT would provide this capability. (OSIRIS on Keck is the current
best option, even though the field of view is a bit too small and the
current AO system at Keck has relatively low strehl at 1$\mu$m. The Keck
system will be upgraded, but TMT should be much better).
Similar data would be needed for the compound lenses.

While these targeted observations would be narrow field, they would
enable some considerable ancillary science, notably  in the areas of
dark matter substructure (from perturbations to the imaged rings) and
AGN host galaxy structure.

% ----------------------------------------------------------------------

\subsection{Time Delay Cosmography: Additional Monitoring of Selected Time Delay Systems}
{\it Phil Marshall, Curtis McCully, Eric Linder and others}

If LSST's cadence is insufficient to provide Stage IV-accurate
measurements of lens time delays, we may need to supplement the light
curves with additional monitoring data. Most of the sample will have
lensed sources that are around 23rd magnitude in brightness, and so
monitoring will require dedicated scheduled imaging effort on 4--10m
class telescopes. For AGN, several year campaigns, with
weekly cadence interleaved with the LSST observations, would be needed.

Given the observational demands,
smaller scale monitoring campaigns at these depths may well be
both desirable as well as feasible. High value small sub-samples might include
lensed type Ia supernovae, and variable sources (i.e.\ AGN or
supernovae) in compound lenses.
The latter may only occur in a handful of systems, but
these highly constrained systems would make valuable targets. Another
possibility currently being explored by Courbin et al, McCully et al,
and others is to carry out dedicated high cadence monitoring of lensed
AGN, using the very short time scale variability to probe the time delay
more efficiently. If this technique can be shown to work, it could
reduce the demands on the monitoring network considerably.

% ----------------------------------------------------------------------

\subsection{Time Delay Cosmography: Multi-object Spectroscopy to Improve the Fidelity of our Lens Environment and Line of Sight Characterization}
{\it Curtis McCully, Adam Bolton, Phil Marshall and others}

LSST will provide photometric redshifts and stellar masses of all
galaxies in the fields of each cosmographic lens. This may not be enough
to allow us to characterize the lens environments and line of sight mass
distribution with sufficient accuracy for Stage IV cosmology.

Characterization of weak lensing effects for strong lensing has
similar needs to that needed to model intrinsic alignments. Multi-object
spectroscopic data can provide significant improvements in
the accuracy of forward models, reducing the residual 
bias in external convergence estimates
by a factor of two CITE{CollettEtal2013}.
Spectroscopic coverage of the $~10$~arcminute strong lens fields
themselves is valuable, where most of the benefits come from the redshifts of
massive galaxies and groups close to the line of sight. DESI data is
potentially good enough in the overlap region with LSST;
photo-z training and calibration fields could provide further coverage.


% ----------------------------------------------------------------------

\subsection{Cluster Mass Function: Spectroscopic Surveying in Cluster Fields}
\label{sec:sl:clusters}
{\it Will Dawson and others}

{\it Note from Will: While I have done a fair amount of cluster weak lensing and
spectroscopy work, this has been focused on dark matter science, I am not
actually an expert in cluster mass function cosmological constraints. That said
I have drafted some text that I hope will describe the methods and needed
capability (or capabilities) in enough detail that someone could determine
whether they overlap with other needs. I am also includeing all spectroscopic
cluster studies and those limited to strong lensing; we can move the text later
if we like.}

{\it Basic Premis}: Cosmology is sensitive to the abundance of rare massive
clusters observed out to and beyond $z\sim 1$. For example, Takada \& Spergel
(2014) have shown that a characterization of the most massive clusters can
improve the constraining power of future cosmic shear surveys (e.g., LSST, and
WFIRST) by approximately a factor of two, provided the cluster masses and
redshifts are accurately constrained.

Spectroscopy of cluster galaxies can improve LSST cluster mass function
constraints in a number of ways:
\begin{itemize}
  \item {\it Redshifts of multiply imaged background galaxies}: Combined strong
  and weak lensing measurements of clusters can improve both the cluster mass
  and concentration constraints by approximately a factor of two (Umetsu et
  al.~2010). So it is possible to improve the LSST cluster weak lensing
  constraints by combining strong lensing measurement of the strongly lensed
  background galaxies (preseumably identified from space-based observations,
  such as HST or WFIRST, but potentially from LSST alone). However, redshifts of
  the multiply lensed galaxies are needed in order to properly constrain the
  cluster mass distribution. There has been some work recently by Keren Sharon's
  group on the number of multiply lensed galaxy redshifts that are necessary for
  a given lens model accuracy. Most of the multiply lensed galaxies are faint $R
  \gtrsim 23$ and require large diameter telescopes.
  \item {\it Accurate cluster redshifts}: For cosmological constraints it is
  necessary to know both the mass and redshift of the clusters. While it is
  possible to obtain better than average photometric redshifts of clusters due
  to their being hundreds to thousands of galaxies to beat down the Poission
  photo-z noise and the tight red sequence color relation, constaints can be
  improved (how much) with accurate redshifts from spectroscopy. It is possible
  to accomplish this with one to a few redshifts of the brightest cluster
  galaxies. The instrument requirements are pretty relaxed for this science
  since on only needs a few redshifts per pointing of rather bright galaxies.
  \item {\it Cluster galaxy velocities as a probe of the cluster potential}: The
  spectroscopic caustic method (Kaiser 1987; Regos \& Geller 1989) is the only
  other successful stand alone method, in addition to weak lenising, of probing
  the mass density profile to comparably large cluster radii. This is important
  since joint CMB+SZE constraints indicate that SZE-derived {\em Planck} masses
  underestimate the true $M_{500}$ masses by $b=1-\langle
  M_{Planck}/M_\mathrm{true}\rangle\sim 43\%$ $\sim43\%$ --- far from the $\sim
  20\%$ expected due to deviations from the influence of feedback, non-thermal
  pressure, and cluster shapes (Battaglia+ 2012). Of even greater concern is
  that according to WL studies (Sereno \& Ettori 2015), the bias appears to vary
  with mass. SZE mass estimates are calibrated in part based on X-ray
  temperatures with spectroscopy that is only reliable out to $r_{2500}\sim
  0.25r_\mathrm{vir}$. Recent WL analyses of the {\em Planck} sample, that probe
  the mass distribution to larger radii, yield $1-\langle
  M_{Planck}/M_\mathrm{WL}\rangle\sim 30\%$ at the $\sim 10\%$ uncertainty level
  (von der Linden+ 2014; Hoekstra+ 2015), somewhat relaxing the {\em Planck}
  tension. Hinting at the importance of having probes of the cluster potential
  out to large radii ($\sim$ the virial radius). To realize this science would
  require more than a hundred spectroscopic redshifts per cluster, with a
  moderate resolution spectrograph. A highly multiplexed moderate resolution
  spectrograph, that can simulatinously target objects within $\lesssim
  5$\,arcsecons of one another (due to the dense cluster environment). I (wd) am
  not sure how many clusters would need to be surveyed to make a significant
  improvement to planned LSST cluster cosmology.
  \item {\it Redshift space distortions}: This is a probe of modified gravity,
  and similar to the spectroscopic caustic method requires hundreds of
  spectroscpic redshifts of cluster galaxies. The spectroscopic demands are the
  same as the caustic method as well. I (wd) am not sure how many clusters we
  would need to make this measurement for.
\end{itemize}

\ldots

% ======================================================================



%D) LSS: BAO (including cross-correlation enhancements), redshift-space distortions on cluster or large scales, QSO absorber power spectra \& BAO
% EG, EJ, WD, JN

%E) Science-driven needs for other resources: software, computing and data management resources, access to archival data, etc.
% AB, CS, EG, WD, BW

\chapter{Science Case 2: Computing, Statistical, and Data Processing Needs}


\section{Data \& Computing Resources}
\label{sec:datacomp}

{\it Necessary resources for data and computing, beyond core LSST project deliverables, to achieve LSST cosmological science goals. Adam Bolton and others.}

The dark-energy science mission of LSST imposes numerous requirements on 
data resources and computing infrastructure for its success. Many of these requirements 
may be beyond the current baseline for LSST construction 
and operations, and some may also be beyond capabilities currently 
anticipated within the DOE-supported LSST Dark-Energy Science Collaboration (DESC)\@.
This section describes these required capabilities as they flow from dark-energy 
science requirements.

\subsection{Data-set access and definition}

The analysis of large LSST datasets for derivation of dark-energy parameter constraints 
will involve development and application of analysis codes by large teams of 
scientists working over periods which are likely to extend over many years, and which will 
require long-term stability and accessibility for scientific reproducibility. This translates 
into requirements that LSST's released static-sky datasets be: 

\begin{itemize}
\item Version controlled
\item Exposed through machine-readable programmatic interfaces
\item Maintained on accessible storage resources for many-year timescales
\end{itemize}

\subsection{Combination of LSST data with other data sets}

Multiple dark-energy science analyses of LSST data will rely on the combination 
of LSST data with redshifts and object parameters from other surveys (both imaging 
and spectroscopic). These analyses impose requirements for: 

\begin{itemize}
\item Long-term access to curated copies of auxiliary datasets
\item Computing infrastructure sufficient for co-analysis of LSST data with auxiliary data
\end{itemize}

\subsection{Likelihood characterization}

Cosmological analyses of LSST data sets will require accurate physical characterization of 
the populations of tracer objects used. This will be necessary both to understand the 
selection-functions of the samples themselves and their evolution with redshift, 
and to connect with physical models for the bias recipes through which these 
populations are related to the underlying dark-matter distribution. In many instances, the 
salient physical parameters of tracers in LSST data may be measured at a level of significance 
that does not admit sufficient characterization through simple Gaussian errors or covariances. 
Such analyses will therefore require: 

\begin{itemize}
\item Explicit identification of all LSST catalog-object 
parameters that are relevant as input to cosmological analysis
\item Sufficient characterization of the joint likelihood of these measured 
parameters on an object-by-object basis, either through direct reporting 
within LSST catalogs, or through custom determination using LSST Level 3 
software infrastructure
\end{itemize}

\subsection{Redshift data validation}

Cosmological analyses of LSST data that rely on spectroscopic redshifts will impose requirements on 
the quality of those redshift data in terms of completeness, purity, precision, and accuracy. In the case 
of spectroscopic samples adopted from other surveys, the requirements for LSST analysis may be 
different than those imposed by the primary science case of those other surveys. Therefore, all 
spectroscopic samples used in LSST dark-energy science analysis will require: 

\begin{itemize}
\item Quantitative specification of the science requirements for redshift and parameter measurements
\item Controlled software infrastructure for validating spectroscopic catalogs 
against LSST dark-energy science requirements
\item Software-development infrastructure for the modification of redshift- and parameter-measurement codes, 
for cases in which a failure to meet requirements is traced not to spectroscopic target-selection or 
observing, but rather to the analysis software used for delivering redshifts from the data
\end{itemize}

\subsection{Mock catalogs}

Simulations, and mock catalogs based on those simulations, will be necessary for multiple 
dark-energy science applications of LSST. These include the characterization of covariances in 
large-scale structure measurements across length scales, determination of the statistics of 
line-of-sight density projection effects for strong-lensing systems, and forward modeling 
of statistical estimators as a function of cosmological parameters. These applications will require: 

\begin{itemize}
\item Allocation of sufficient computing and storage resources to conduct suites of cosmological 
$N$-body simulations tailored to LSST dark-energy science analysis
\item Support for the generation, documentation, and application of mock LSST catalogs based on 
appropriate simulations.
\end{itemize}

\subsection{Survey characterization}

Precision cosmological measurements of the baryon acoustic oscillation (BAO) feature using LSST
data will be sensitive to percent-level systematics that may vary across the LSST footprint in a 
complex field-dependent geometric pattern. In order that these variations do not compromise 
the precision and accuracy of the cosmological measurements, the following capabilities 
will be required: 

\begin{itemize}
\item Sufficient analysis within core LSST observing and data-reduction operations to enable 
determination of field-dependent quantities relevant for dark-energy science analysis 
\item Computation of weight-masks for bright stars, stellar number density, effective seeing, and 
effective depth in LSST static-sky data releases, along with a documented and 
supported data-model for the encoding of these masks
\item Software infrastructure for incorporating these masks into LSST dark-energy parameter analysis
\end{itemize}




%F) Development of statistical techniques; e.g. for utilizing photo-z pdf information/samples for, e.g. clustering measurements.  Developing techniques for probabilistic inference over redshift distributions
% CS, EG, JN, WD, BW, EJ


LSST will generate very large galaxy samples that can be used for
investigations of the galaxy distribution -- e.g. correlation
functions, luminosity functions, dependence on environment -- that are
much more sensitive than what we are used to. These will enable
measurement of features and evolution at high precision. At the same
time, these samples will largely rely on photometric redshifts. To
take full advantage, and to avoid introducing subtle systematics,
astronomers will have to develop methods that use the full information
in the photo-z P(z) estimate, rather than using point estimates
(placing galaxies at their best estimate redshift).

This will require developing techniques for probabilistic inference
over photo-z probability distributions, when measuring typical
population estimates such as the luminosity function or correlation
function. It will generally not be practical to forward-model the
entire dataset (e.g. constraining the luminosity function directly
from galaxy apparent magnitudes, in principle would require modeling
the LF and colors of all galaxies at all redshifts). So investigators
will use photo-z probability distributions calculated either by the
LSST pipeline or their own.  Generalizing the standard measurements to
use the information in the P(z) distribution and to be robust against
errors in estimating P(z) is a relatively unexplored area. Simple
ideas such as randomly sampling from each galaxy's P(z) can give wrong
answers (because this magnifies photo-z spread rather than
compensating for it). Exploiting the LSST dataset will require
supporting research into methods for inference using photo-zs and
supporting the community in developing and releasing publicly
available code and tools.





 \end{document}
