% ======================================================================

\section{Strong Lensing}
\label{sec:sl}

% {\em C) Strong lensing: Monitoring, spectroscopy, positions (incl. IFU
% spectroscopy, monitoring of lens solution for supernovae, high
% resolution imaging follow-up with ELTs, spectroscopy to enable combined
% SL/WL analysis of clusters).}

% For each science area, we should:
%
% 1) Describe the science need in a few sentences at most
%
% 2) Describe the needed capability (or capabilities) in enough detail
% that someone could determine whether they overlap with other needs.
% (E.g.:  “medium-resolution (R~4000-6000) spectroscopy covering the
% full optical window for i<25.3 objects with multiplexing of ~1000 over
% ~10 arcminute diameter fields” could work, or for a differentcase
% “optical medium-resolution spectroscopy with a highly-multiplexed
% spectrograph with a many-degree FoV on a 4m telescope” would also give
% enough information to allow identification of common needs).

% ----------------------------------------------------------------------

\subsection{Time Delay and Compound Lens Cosmography: High Precision Galaxy Mass Models from High Resolution Imaging and Spectroscopy}
{\it Phil Marshall, Adam Bolton and others}

The primary route to cosmology from strong lensing is time delays in
galaxy-scale lensed quasars and supernovae. Galaxy scale compound lenses
(i.e. systems with two sources at different redshifts) have also been
suggested.
To be useful as probes of cosmological distances, galaxy scale lenses
need very well constrained mass models. These constraints will come from
a) high resolution imaging of the Einstein rings due to the source
AGN or SN host galaxy, and b) spatially resolved spectroscopy of the lens
galaxy, enabling measurement of the stellar velocity dispersion field.

% We now describe the needed capability (or capabilities) in enough detail
% that someone could determine whether they overlap with other needs.

We expect to be able to compile samples of several hundred lensed AGN
and  lensed SN systems with accurately measured LSST time delays (CITE
Liao et al 2015). Targeted snapshot (i.e.\ 200--2000 second exposure time)
imaging in the optical or near infra-red (depending on the system)
with either JWST or GSMTs (for the fainter sytems) or next generation AO
on 10-m class telescopes (for the brighter ones) can provide
the Einstein ring constraints needed to turn each of these systems into
a  5\% precision distance CITE{MengEtal2015}. An accurate spectroscopic
velocity dispersion is also needed, to break the degeneracy between the
predicted time delays and the lens mass density profile:
deeper data from the same facilities are
also likely to be needed for these 18-21$^{\rm st}$ magnitude galaxies.
For lens modeling, the fields of view can be
as small as a few arcseconds, provided the PSF is well characterized.

Similar data would be needed for the compound
lenses: here, we also  aspire to a sample of $\sim 100$ systems. While
these targeted observations would be narrow field, they would enable
some considerable ancillary science, notably  in the areas of dark
matter substructure (from perturbations to the imaged rings) and AGN
host galaxy structure. Source redshifts may  need to come from targeted
observations with 10-m class telescopes  before going to the larger
facilities: this is itself a large program.


% ----------------------------------------------------------------------

\subsection{Time Delay Cosmography: Additional Monitoring of Selected Time Delay Systems}
{\it Phil Marshall, Curtis McCully, Eric Linder and others}

If LSST's cadence is insufficient to provide Stage IV-accurate
measurements of lens time delays, we may need to supplement the light
curves with additional monitoring data. Most of the sample will have
lensed sources that are around 23rd magnitude in brightness, and so
monitoring will require dedicated scheduled imaging effort on 4--10m
class telescopes. For AGN, several year campaigns, with
weekly cadence interleaved with the LSST observations, would be needed.

Given the observational demands,
smaller scale monitoring campaigns at these depths may well be
both desirable as well as feasible. High value small sub-samples might include
lensed type Ia supernovae, and variable sources (i.e.\ AGN or
supernovae) in compound lenses.
The latter may only occur in a handful of systems, but
these highly constrained systems would make valuable targets. Another
possibility currently being explored by Courbin et al, McCully et al,
and others is to carry out dedicated high cadence monitoring of lensed
AGN, using the very short time scale variability to probe the time delay
more efficiently. If this technique can be shown to work, it could
reduce the demands on the monitoring network considerably.

% ----------------------------------------------------------------------

\subsection{Time Delay Cosmography: Multi-object Spectroscopy to Improve the Fidelity of our Lens Environment and Line of Sight Characterization}
{\it Curtis McCully, Adam Bolton, Phil Marshall and others}

LSST will provide photometric redshifts and stellar masses of all
galaxies in the fields of each cosmographic lens. This may not be enough
to allow us to characterize the lens environments and line of sight mass
distribution with sufficient accuracy for Stage IV cosmology.

Characterization of weak lensing effects for strong lensing has
similar needs to that needed to model intrinsic alignments. Multi-object
spectroscopic data can provide significant improvements in
the accuracy of forward models CITE{CollettEtal2013}.
Spectroscopic coverage of the $~10$~arcminute strong lens fields
themselves is valuable, where most of the benefits come from the redshifts of
massive galaxies and groups close to the line of sight. DESI data is
potentially good enough in the overlap region with LSST;
photo-z training and calibration fields could provide further coverage.


% ----------------------------------------------------------------------

\subsection{Cluster Mass Function: Spectroscopic Surveying in Cluster Fields}
\label{sec:sl:clusters}
{\it Will Dawson and others}

{\it Note from Will: While I have done a fair amount of cluster weak lensing and
spectroscopy work, this has been focused on dark matter science, I am not
actually an expert in cluster mass function cosmological constraints. That said
I have drafted some text that I hope will describe the methods and needed
capability (or capabilities) in enough detail that someone could determine
whether they overlap with other needs. I am also includeing all spectroscopic
cluster studies and those limited to strong lensing; we can move the text later
if we like.}

{\it Basic Premis}: Cosmology is sensitive to the abundance of rare massive
clusters observed out to and beyond $z\sim 1$. For example, Takada \& Spergel
(2014) have shown that a characterization of the most massive clusters can
improve the constraining power of future cosmic shear surveys (e.g., LSST, and
WFIRST) by approximately a factor of two, provided the cluster masses and
redshifts are accurately constrained.

Spectroscopy of cluster galaxies can improve LSST cluster mass function
constraints in a number of ways:
\begin{itemize}
  \item {\it Redshifts of multiply imaged background galaxies}: Combined strong
  and weak lensing measurements of clusters can improve both the cluster mass
  and concentration constraints by approximately a factor of two (Umetsu et
  al.~2010). So it is possible to improve the LSST cluster weak lensing
  constraints by combining strong lensing measurement of the strongly lensed
  background galaxies (preseumably identified from space-based observations,
  such as HST or WFIRST, but potentially from LSST alone). However, redshifts of
  the multiply lensed galaxies are needed in order to properly constrain the
  cluster mass distribution. There has been some work recently by Keren Sharon's
  group on the number of multiply lensed galaxy redshifts that are necessary for
  a given lens model accuracy. Most of the multiply lensed galaxies are faint $R
  \gtrsim 23$ and require large diameter telescopes.
  \item {\it Accurate cluster redshifts}: For cosmological constraints it is
  necessary to know both the mass and redshift of the clusters. While it is
  possible to obtain better than average photometric redshifts of clusters due
  to their being hundreds to thousands of galaxies to beat down the Poission
  photo-z noise and the tight red sequence color relation, constaints can be
  improved (how much) with accurate redshifts from spectroscopy. It is possible
  to accomplish this with one to a few redshifts of the brightest cluster
  galaxies. The instrument requirements are pretty relaxed for this science
  since on only needs a few redshifts per pointing of rather bright galaxies.
  \item {\it Cluster galaxy velocities as a probe of the cluster potential}: The
  spectroscopic caustic method (Kaiser 1987; Regos \& Geller 1989) is the only
  other successful stand alone method, in addition to weak lenising, of probing
  the mass density profile to comparably large cluster radii. This is important
  since joint CMB+SZE constraints indicate that SZE-derived {\em Planck} masses
  underestimate the true $M_{500}$ masses by $b=1-\langle
  M_{Planck}/M_\mathrm{true}\rangle\sim 43\%$ $\sim43\%$ --- far from the $\sim
  20\%$ expected due to deviations from the influence of feedback, non-thermal
  pressure, and cluster shapes (Battaglia+ 2012). Of even greater concern is
  that according to WL studies (Sereno \& Ettori 2015), the bias appears to vary
  with mass. SZE mass estimates are calibrated in part based on X-ray
  temperatures with spectroscopy that is only reliable out to $r_{2500}\sim
  0.25r_\mathrm{vir}$. Recent WL analyses of the {\em Planck} sample, that probe
  the mass distribution to larger radii, yield $1-\langle
  M_{Planck}/M_\mathrm{WL}\rangle\sim 30\%$ at the $\sim 10\%$ uncertainty level
  (von der Linden+ 2014; Hoekstra+ 2015), somewhat relaxing the {\em Planck}
  tension. Hinting at the importance of having probes of the cluster potential
  out to large radii ($\sim$ the virial radius). To realize this science would
  require more than a hundred spectroscopic redshifts per cluster, with a
  moderate resolution spectrograph. A highly multiplexed moderate resolution
  spectrograph, that can simulatinously target objects within $\lesssim
  5$\,arcsecons of one another (due to the dense cluster environment). I (wd) am
  not sure how many clusters would need to be surveyed to make a significant
  improvement to planned LSST cluster cosmology.
  \item {\it Redshift space distortions}: This is a probe of modified gravity,
  and similar to the spectroscopic caustic method requires hundreds of
  spectroscpic redshifts of cluster galaxies. The spectroscopic demands are the
  same as the caustic method as well. I (wd) am not sure how many clusters we
  would need to make this measurement for.
\end{itemize}

\ldots

% ======================================================================
