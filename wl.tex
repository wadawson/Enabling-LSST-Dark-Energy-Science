\section{Weak lensing}

{\em Draft notes from Rachel, to be reviewed by Will and iterated on before the rest of the group
  takes them too seriously.}

\subsection{Galaxy morphologies}

All existing methods of inferring weak lensing shear depend in some way on knowledge of (at minimum)
the galaxy intrinsic ellipticity distribution, and in many cases they also depend on the
distribution of higher-order moments of galaxies as well.  Do we require additional observations
from US OIR facilities to obtain that information for LSST?  I (RM) had raised this question on the
telecon, but here are my current thoughts now that I consider the question in more detail.  I think
the answer depends on the method of shear inference:

\begin{itemize}
\item Regular old ``measure shapes and average to infer ensemble shear'' methods: these typically
  have implicit assumptions about the distributions of galaxy morphologies and ellipticities, which
  should be calibrated out using simulated data that has a realistic distribution of all those
  things.  Existing efforts to do this rely on HST data.  I would argue that we need more HST data
  to avoid being cosmic variance dominated for the very faintest galaxies (currently only observed
  in UDF), but that is outside the scope of this committee's efforts.  Alternatively, I think that
  the best-seeing subset of LSST data may provide some information on this, and it's unlikely that
  observations with some other telecope could provide a comparable volume of high-quality data.
\item Bayesian Fourier approaches (Bernstein \& Armstrong): this approach requires a deep (high
  $S/N$) dataset to place priors on the distributions of galaxy moments.  Their proposal is to use a
  deep subset of the same dataset used for the science; in the LSST context, this would mean using
  the DDFs.  No external data needed.
\item Hierarchical inference (Schneider et al.): A prior on the intrinsic ellipticity distribution
  has to come from somewhere.  Presumably the source could be the same as for (1), not requiring
  additional US OIR facilities outside of LSST.  It's not clear how information about detailed
  morphologies would go into this approach, but perhaps at the stage of testing calibration using
  simulations.  In this case the needs are similar to those of (1) as well.
\end{itemize}

Tentative conclusion: our report doesn't have to say anything about calibrations of galaxy
morphology and ellipticity distributions, because we will get what we need either from LSST data
itself or from HST.

\subsection{Intrinsic alignments}

