\section{Weak lensing}
\label{sec:wl}

{\it B) Weak lensing: Intrinsic alignment studies, exquisite-seeing data
for morphology templates, etc.}

\subsection{Galaxy morphologies}

After some discussion, we conclude that our report doesn't have to say anything about calibrations
of galaxy morphology and ellipticity distributions, because we will get what we need either from
LSST data itself or from HST.  (For longer discussion of why this is the case, see the version of
this document in commit b29e99f64e6633218f3316092b6d6fdb38f6d209.)

\subsection{Intrinsic alignments}

Intrinsic alignments (IA) of galaxy shapes with the cosmic web are a contaminant to weak lensing
measurements, since WL measurements assume that all coherent alignments are due to lensing. IA have
been robustly detected out to hundred Mpc scales, making them a serious theoretical uncertainty for
WL cosmology contaminating both density-shape and shape-shape correlation functions.  Several
methods have been developed for mitigating this systematic, including nulling (which loses a lot of
cosmological information and leads to stringent constraints on photo-$z$ errors), forward modeling
(which uses galaxy clustering, galaxy-shear, and shear-shear correlations measurements, and involves
marginalizing over a model for how the alignments enter those measurements), and self-calibration
(which does not require an {\em a priori} alignments model, but has certain assumptions that have
not yet been validated).

Current data, primarily from the SDSS, provide us a template for how intrinsic alignments scale with
galaxy type, luminosity, and redshift for $z\lesssim 0.5$.  However, we need this information for
galaxies at higher redshifts, and we need better constraints on the alignments of blue galaxies, in
order to place reasonable priors when carrying out the WL analysis with LSST. Generally, this
requires both good imaging and redshift information, in order to better localize the galaxy pairs in
3D and measure galaxy shapes.  Thus, the LSST data itself will ultimately be quite informative about
IA despite the size of the photo-$z$ errors, but there is still value in strong external priors on
the IA model because IA are degenerate with other systematics (photo-$z$ errors and certain shear
systematics).

In principle a direct measurement of IA requires a spectroscopic dataset that covers decent-sized
contiguous areas (so as to constrain alignments to tens of Mpc) but also has a decent sampling rate
of a {\em representative galaxy sample} within these fields.  This seems somewhat orthogonal to a
``many small fields to beat down cosmic variance'' approach that one might want for spectroscopic
observations to constrain photo-$z$ errors, but perhaps a compromise could be reached.  Another
question is whether we could tolerate a more sparse and/or unrepresentative sampling of the density
field over a wider area, and constrain IA using cross-correlations with the spectroscopic galaxies
instead of the auto-correlations of them (using a method like that from Blazek et al 2012 or Chisari
et al 2014).  We will explore what compromises can be made while still supporting the LSST dark
energy science at the needed level in the coming weeks.
