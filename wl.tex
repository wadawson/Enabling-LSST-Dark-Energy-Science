\section{Weak lensing}
\label{sec:wl}

A supporting science case that could use an identical (or very similar) dataset to those described
in Section~\ref{sec:design} comes from the area of weak lensing.  Multi-object spectroscopy can be
used to help mitigate a major theoretical systematic for weak lensing: intrinsic alignments of galaxy
shapes.

Intrinsic alignments (IA) of galaxy shapes with the cosmic web are a contaminant to weak lensing
measurements, since WL measurements assume that all coherent alignments are due to lensing. IA have
been robustly detected out to hundred Mpc scales, making them a serious theoretical uncertainty for
WL cosmology contaminating both density-shape and shape-shape correlation functions.  Several
methods have been developed for mitigating this systematic, including nulling (which loses a lot of
cosmological information and leads to stringent constraints on photo-$z$ errors), forward modeling
(which uses galaxy clustering, galaxy-shear, and shear-shear correlations measurements, and involves
marginalizing over a model for how the alignments enter those measurements), and self-calibration
(which does not require an {\em a priori} alignments model, but has certain assumptions that have
not yet been validated).  Given the limitations of all these methods, it is important also to
explore intrinsic alignment effects directly, which requires spectroscopic redshifts to determine
which galaxies are in physical proximity to each other.  This direct exploration will provide
intrinsic alignments models as inputs to the mitigation methods that need models; it requires low
resolution spectroscopy or spectro-photometric redshifts similar to those from the PRIMUS
survey, but potentially for a deeper sample of galaxies.
% JAN: I added that last sentence in response to a question from Joan Najita:
%How critical is the spectroscopy? The first paragraph mentions several methods -- do they all need spectroscopy? If not, how much does the spectroscopy add to the science result? (Is it "nice to have" or "critical"?)


Current data, primarily from the SDSS, provide us a template for how intrinsic alignments scale with
galaxy type, luminosity, and redshift for $z\lesssim 0.5$.  However, we need this information for
galaxies at higher redshifts, and we need better constraints on the alignments of blue galaxies, in
order to place reasonable priors when carrying out the WL analysis with LSST. Generally, this
requires both good imaging and redshift information, in order to better localize the galaxy pairs in
3D and measure galaxy shapes.  Thus, the LSST data itself will ultimately be quite informative about
IA despite the size of the photo-$z$ errors, but there is still value in strong external priors on
the IA model using spectroscopic redshifts, because IA are degenerate with other systematics (photo-$z$ errors and certain shear
systematics).

In principle a direct measurement of IA requires a spectroscopic or spectro-photometric dataset that covers decent-sized
contiguous areas (so as to constrain alignments to tens of Mpc) but also has a decent sampling rate
of a {\em representative galaxy sample} within these fields.  This seems somewhat orthogonal to a
``many small fields to beat down cosmic variance'' approach as in the ``training'' dataset described in
Sec.~\ref{sec:design}), but perhaps a compromise could be reached. For example, with a field
size of 1 degree, at $z\sim 0.8$ this corresponds to $\sim 25$Mpc, which should be helpful in
constraining the IA model on the scales below $\sim 10$Mpc where existing theoretical models are
most uncertain. Another question is whether
we could tolerate a more sparse and/or unrepresentative sampling of the density field over a wider
area, and constrain IA using cross-correlations with the spectroscopic galaxies instead of the
auto-correlations of them (using a method like that from Blazek et al 2012 or Chisari et al 2014).
The dataset needed for this could well look like  the ``calibration'' dataset for enabling the
photo-$z$ cross-correlation method in Sec.~\ref{sec:design}. More exploration is needed to determine
precise survey design requirements for this supporting science case, but we anticipate they will be
reasonably close to what is in Sec.~\ref{sec:design}.

