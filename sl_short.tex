\subsection{Science Goals}

Describe your science goals here, as in the scientific justification of an NOAO proposal. If you have multiple science goals, you can either describe them all here, or replicate the science, technical, and capabilities sections for each goal. Just create one summary table for the entire program.  

%NOAO guidelines:
%
%The scientific justification should explain the overall goals of
%your program in the context of your field, as well as the importance
%of your program to astronomy.
%Writing a good scientific justification is an art.  It takes
%skill and practice.  And it requires a good scientific idea.
%This last you must supply but a few general guidelines
%about proposal writing might still be helpful...
%
%\begin{itemize}
%\item
%State succinctly and clearly the problem you are trying to solve
%and the progress that will be made toward doing so if the proposed
%observations are successful.  If the review panel members have to work hard
%even to understand what you want to do, they are unlikely to be
%sympathetic to your proposal.
%
%\item
%Explain clearly why the project is important and how it
%relates to the broad context and important issues in your field.
%Many proposals focus too tightly on a specific observational
%goal (e.g. ``measure the velocity dispersion of this cluster of galaxies'')
%without explaining why it is important or how it relates to a
%significant question about the Universe.
%
%\item
%Be specific.  If your observations will ``constrain theoretical
%models,'' then discuss what will be constrained and why those
%constraints matter.  Make sure the review panel understands exactly why
%the observations you propose will make a difference in your field,
%and exactly how the observations will refine or
%require changes in the theory.
%
%\item
%Keep it simple.  Try to focus on the central idea of your proposal.
%Complex arguments are hard to explain and hard for the panel members to follow.
%Distracting tangential arguments obscure the theme of your proposal.
%
%\item
%Include a figure to help explain what you want to do.  Sample
%data or model predictions shown in a figure often help clarify
%complex arguments for the panel members.
%
%\item
%Keep it short.  Never exceed a page for the text of the scientific
%justification, and never reduce the font size.  It may even help to
%be a little under a page, and increase the font size a little!
%Organize your presentation with paragraphs, headings, and bullets
%so it is easy to read.



\subsection{Technical Description }

Give a technical description of your program, describing e.g., sample size and properties, justification of spectral or spatial resolution, wavelength, target density, etc.


%NOAO guidelines:
%The review panel looks to this section to find out about the overall
%strategy of your observational program.  Why do you need the telescopes
%and instruments you request? How are your targets selected?
%Why do you need spectroscopy or imaging, and what measurements will
%you make from the data?  Why is your approach to be preferred over
%some other approach, what must the minimum sample size be to achieve
%your scientific goals (and why), and why are your
%observations likely to be better than previous work in the field?



\subsection{Needed Capabilities and Estimate of Demand}

Describe the needed capabilities and demand (e.g., estimate of observing time) that flow down from the science and technical considerations. If applicable, describe the time critical nature of the required capabilities (do you need to have the capabilities while LSST is carrying out the survey or can you do the follow up later?) 


