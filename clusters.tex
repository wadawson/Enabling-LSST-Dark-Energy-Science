% ======================================================================

\subsection{Galaxy Clusters}
\label{sec:clusters}
{\it Anja von der Linden, Will Dawson, Elise Jennings and others}

The abundance of massive clusters of galaxies is highly sensitive to cosmological parameters as it is closely related to the abundance of massive dark matter halos.  Small changes in the rate of growth of dark matter overdensities (which, if general relativity is correct, is a simple function of basic cosmological parameters) will yield exponential changes in the mass function of halos at the high-mass end.  As a result, measurements of the cluster mass function will be a statistically very powerful technique in the LSST era, with samples of tens to hundreds of thousands of clusters over a wide range in redshift.  However, harnessing this statistical power to constrain cosmology accurately requires a detailed understanding of all the systematic uncertainties involved.  

The dominant sources of systematic uncertainties relate to how well we can understand the relation between the observables captured by different measurement methods -- including cluster richness from LSST-based cluster finders, the Sunyaev-Zeldovich signal in AdvACT/SPT-3G/CMB-S4 mm-wavelength observations, and X-ray flux from space observations (e.g., eROSITA) -- and the true virial masses of the clusters' host dark matter halos.  LSST is uniquely positioned to address the calibration of mass-observable relations due to its weak-lensing and photo-$z$ capabilities, which can be used for an accurate and precise calibration of cluster masses (DESC 2012, 2015).  The needs for cluster cosmology in terms of additional ground-based OIR data are therefore intricately linked with those of weak lensing and photometric redshifts.  The largest OIR effort we foresee that would be primarily driven by cluster cosmology is training and tests of photometric redshift algorithms in extreme environments.  This work would require instrumentation equivalent to that required for photometric redshift training and calibration (cf. \S \ref{sec:design}).

% ----------------------------------------------------------------------

\subsubsection{Optimizing Photometric Redshifts in Cluster Fields}
\label{sub:cluster_photz}

Photo-$z$'s in cluster fields are necessary for two purposes: both to measure the cluster redshift and to cleanly separate background galaxies which can be used in the weak lensing measurements from cluster and foreground galaxies.  The latter dominates empirical uncertainties in weak lensing-based cluster mass estimates, and is thus the primary driver for obtaining additional data to optimize cluster count cosmology with LSST.  The required datasets have similar instrumental requirements as the photometric redshift training samples described in \S \ref{sec:design}.

%\begin{itemize}
%\item {\it Photo-z testing in cluster fields:} 
Since the shear induced on a background galaxy is a function of the galaxy redshift, the photo-$z$'s of galaxies in cluster fields need to be accurately known.  It has been shown that while using only simple one-point estimators of photometric redshift (e.g., the peak or weighted mean of the posterior probability distribution) leads to significantly biased results, incorporating the full photo-z redshift probability distribution $p(z)$ into analyses greatly decreases biases in mass measurements (Applegate et al. 2014).  In addition to general requirements on characterizing $p(z)$ distributions (cf. the calibration requirements in \S \ref{sec:photoz_just}), for some analyses the distributions used should accurately reflect the probability a given galaxy lies within a cluster.  

Effectively, one wishes to incorporate a prior that depends upon location on the sky.  Galaxies along a line of sight that contains a cluster are more likely to be at the cluster redshift than would be expected based upon posterior probability distributions calculated without any reference to position.   The strength of the modulation of the prior should depend upon both cluster redshift and mass (or richness).  It may be possible to characterize the cluster populations via cross-correlation methods (the total population of cluster galaxies can be characterized using the cluster/photometric galaxy cross-correlation, cf. Phillips XYZ).  Models for this modulation can be tested (or, if cross-correlation methods prove infeasible, developed) by obtaining spectroscopic redshifts for galaxies in a set of clusters; spectroscopy of $\sim 100--300$ galaxies in each of $\sim 20--50$ clusters would be useful for this purpose.  Instrumental requirements for this work are very similar to those for photometric redshift training (cf. \S \ref{sec:design}), though field-of-view and multiplexing requirements are weaker (given that the relevant spatial scale is the cluster size, $\sim 1-2$ Mpc physical).  This project could therefore be pursued efficiently on the smaller spectrographs expected for GSMTs.  It is likely that methods of incorporating cluster presence in priors could be tested with a subset of brighter galaxies (e.g., those with $i<24.3$ instead of $i<25.3$) rather than going down to the LSST weak lensing magnitude limit, so exposure times per cluster could be considerably less than the per-field exposure times calculated in \S \ref{sec:design}.
%{\it AvdL comment: need to figure out prospects for cross-correlation methods for clusters}
%\end{itemize}


%JAN I'm commenting the following out -- it would be difficult to cover wide areas from the ground to even Euclid depth, let alone WFIRST (and at $z>1$ the red sequence stops being a universal feature of clusters, resulting in significant incompleteness in this domain).
%
%In terms of determining cluster redshifts, techniques based on red-sequence galaxies have been shown to perform exceptionally well at least to at least $z\le0.8$ (Rykoff et al. 2016), and are expected to yield reliable redshifts to $z\le1.2$ with LSST. 
%\begin{itemize}
%\item {\it Cluster redshifts at $z\ge1.1$:} Red-sequence-based methods recover accurate cluster redshifts if the red sequence is (1) already well-established, and (2) falls within two bands used in the survey. At higher redshifts, neither of these is necessarily the case.  Yet, the high-redshift clusters are particularly valuable for probing dark energy.  NIR imaging, as well as follow-up spectroscopy, can significantly improve the cluster redshift estimates in this regime.  Gathering such datasets can be done already before completion of LSST by targeting SZ-selected cluster samples.  
%%{\it AvdL comment: Euclid and WFIRST can probably do a lot of this?})
%\end{itemize}
%
\subsubsection{Testing alternative theories of gravity}

The combination of weak lensing with dynamical mass probes (e.g., measurements of galaxy velocities within clusters) can be used to test non-GR theories of gravity, which could provide an alternative explanation for the cosmic acceleration commonly described to dark energy.   A number of models would modify the way in which the lensing potential depends on matter density perturbations, such as nonlocal gravity, Galileon gravity, massive gravity, Kmouflage gravity, and several other subclasses of Horndeski's general theory.  

These modifications to the lensing signal give rise to a broad range of ways to test gravity.  For instance, models that change the lensing signal can also have a strong impact on the power spectrum of cosmic shear and the lensing of cosmic microwave background (CMB) photons.  
As described in Zu et al. 2014, the combination of measurements of galaxy
infall motion around massive clusters  with weak lensing measurements offers
a powerful non-parametric cosmological test of modified gravity.  As an example of the sorts of analyses possible, Barreria et al. 2015 found that for Galileon models, lensing masses of clusters are unchanged as non-GR effects are screened where curvature is larger, but the force profile of clusters can be 10 -- 40\% higher than in $\Lambda$CDM at distances $\sim 2-20 h^{-1}$ Mpc from the cluster center, affecting galaxy infall velocities in cluster outskirts.   

The same spectroscopy used to characterize photometric redshifts around clusters (described in \S \ref{sub:cluster_photz}) can be used to measure infall velocities and hence provide constraints on the origin of cosmic acceleration, so long as it reaches these distances from cluster centers.  A dedicated survey of brighter galaxies near clusters could enable stronger constraints.  {\it AvdL comment: 
%there are many version of this: redshift-space distortions, stacked measurements, etc.  
How many redshifts per cluster do we actually need?}
%
%practically unmodified lensing masses in the Galileon model can have interesting implications for tests of gravity that are designed to detect differences between lensing and dynamical mass estimates. Although the Galileon model modifies the lensing potential, this does not translate into modified lensing masses because of the gravitational screening imposed. However, outside $R_{200}$ (at distances $\sim 2 --20 h^{-1}$ Mpc from the cluster center), the force profile of clusters in the Galileon model can be 10 -- 40\% higher than in $\Lambda$CDM, which can affect the velocity distribution of infalling galaxies, and hence those dynamical mass estimates which are based on infall velocities.
% 
The main systematic uncertainty in this measurement arises from our imperfect understanding of the impact of galaxy evolution on infall kinematics.  Detailed measurements of the dependence of galaxy clustering on galaxy properties from the survey described in \S XYZ would help to constrain these systematics. 
%and to complement
%WL as a cosmological test of gravity, this modelling of
%galaxy clusters requires overlap with a large galaxy redshift survey,
%such as BOSS, eBOSS, the Subaru Prime Focus
%Spectrograph, DESI or WFIRST.   



% JAN I'm commenting out the voids -- I don't see what observations this would motivate.
%The existence of screening mechanisms in modified gravity models also motivates tests of lensing effects associated with cosmic voids, where density is low and hence screening effects are negligible.  
%% fifth force effects are manifested more prominently. In the context of modified gravity, v
%Voids found in simulations are on average emptier in e.g. $f(R)$ gravity than in $\Lambda$CDM because the enhanced unscreened gravity facilitates the pile up of matter in the surrounding walls and filaments, leaving less mass inside the void. This translates into a stronger lensing signal from voids than would otherwise be expected.

%For a given GR-based cosmology with a smooth dark energy component, a measurement of the expansion history gives a prediction for the growth rate of matter density perturbations. The growth rate can be studied by measuring the clustering of galaxies in redshift space, where peculiar velocities distort the clustering signal along the line of sight. By testing the consistency between the measured growth rate and the prediction from the expansion history, it is possible to constrain models of modified gravity and distinguish them from GR (cf. Zu et al. 2014, Jennings et al. 2012, Wyman et al. 2013, etc.).
%%Many studies have found that redshift space distortions in modified gravity models have an impact on the clustering of dark matter on large and small scales to a level which may be distinguished from general relativity. 
%Current redshift space distortion models, which are valid on quasi linear scales, are accurate enough to extract the non linear matter $P(k)$ in real space from the measured 2D redshift space power spectrum on large scales, allowing us to constrain $f(R)$ modified gravity. The large difference between the predicted velocity P(k) in the $f(R)$ and GR cosmology make this a very promising observable with which to test GR provided that this can be accurately extracted from the redshift space power spectrum. 



\subsubsection{Merging Galaxy Clusters}

Merging galaxy clusters provide one of the best means of constraining the
self-interacting nature of DM particles, which can only be tested
astrophysically. The merging cluster SIDM signature is also very different from
that of galaxy and galaxy cluster density profile constraints, that can be
mimicked by baryonic processes, and provides a larger velocity range to test
velocity dependent SIDM models.

{\it Motivation:} The tightest SIDM constraint comes from dissociative mergers
that are seen soon after first pericenter passage, when the mostly collisionless
galaxies and DM have passed through each other and proceeded ahead of much of
the gas, which formed a central pancake during collision.  If DM self-interacts
it would experience a drag force during the collision and thus lag the galaxies.
This basic picture has been confirmed by simulations (Randall et al. 2008,
Kahlhoefer et al. 2013), and Randall et al. (2008) find an upper limit of
${\sigma_{DM}\over m} < 1.25$ cm$^2$ g$^{-1}$ based on the galaxy-DM offset
(which has a nonzero central value but is consistent with zero) in the Bullet
Cluster.  That limit is a factor of a few from being very interesting
astrophysically because cross sections of 0.1-0.5 would explain a number of
anomalies referenced elsewhere in this report: cluster density profiles (Newman
et al 2013a,b), the absence of high-density Milky Way satellites predicted by
cold dark matter (CDM) simulations (Boylan-Kolchin et al 2011, 2012), the
cusp-core problem (Navarro et al 2004, Walker \& Pe\~{n}arrubia 2011, Wolf \&
Bullock 2012), and other halo shape and profile observations (Peter et al 2013,
Rocha et al 2013).  Furthermore, there are well-motivated particle models which
yield cross sections in this range (Arkani-Hamed et al.\,2009; Feng et
al.\,2010).  SIDM detected at this level would very much break the dominant WIMP
paradigm, which is also looking weaker based on recent collider and
direct-detection results.  However, cored density profiles, satellites with low
central densities, etc., can also be explained by baryonic processes. Using
mergers as DM colliders has the potential to break this degeneracy.

{\it Needs:} SIDM in merging galaxy clusters can be constrainted by measuring
the distribution of the effectively collisionless galaxies relative to the dark
matter distribution. LSST is capable of providing both of these measurements
through photometric selection of cluster galaxies and weak/strong lensing
mapping of the dark matter distribution. However, the The predicted galaxy-DM
offset is a function of cluster surface mass density, collision speed, and time
since pericenter as well as the DM particle model. Thus it is just as important
to model the geometry and dynamical history of the merger (e.g., Dawson 2013).
This requires measuring the masses of the subclusters (to evolve the system back
in time to infer the speed at the time of pericenter and time since pericenter)
and their relative line-of-sight (LOS) velocity (to constrain the 3-d velocity
and projection angle). To accurately measure the relative LOS velocity of the
merging clusters requires and identify potential substructure that can confuse
the interpretation of the merger, requires approximately 100 redshifts per
merging subcluster. Many of the order thousand dissociative merging clusters
discovered by LSST will be at redshifts of $>0.5$. The scale of individual
clusters is $\sim 1$\,Mpc and the separation of subclusters in a dissociative
merger is $\sim 1--5$\,Mpc. To most efficiently survey these mergers requires a
highly multiplexed multi-object spectrograph with a field of view of $\sim
15$\,arcminutes. Furthermore, given the dense cluster environment it is ideal to
be able to simultaneously observe galaxies separated by $\lesssim
5$\,arcseconds. To obtain the needed hundreds of cluster spectra at these
redshifts it will be necessary to survey down to $R\sim23$ for clusters at $z\sim 0.5$.  Hence, the instrumental requirements are very similar to those described in \S \ref{sec:design}, but total time requirements are much reduced due to the comparative brightness of the objects studied.


{\bf  JAN TODO: Incorporate the following into the strong lensing chapter.  This doesn't really belong with MOS.}
\subsubsection{Cluster density profiles}

Clusters with multiple mass probes covering a range in radial range can be used to reconstruct the cluster density profiles, which serves as a validation of the density profiles assumed in the weak-lensing mass measurements, and could be used to test the properties of dark matter, if the astrophysical processes at cluster cores can also be understood (Peter et al. 2013).  In addition to the LSST weak-lensing measurements, which best constrain cluster masses on $\sim1$~Mpc scales, such datasets should include BCG velocity dispersion profiles, spectroscopic redshifts of any strongly lensed multiple image systems, and relatively deep X-ray imaging. In terms of OIR capabilities, this translates to deep long-slit or IFU spectra of the BCG, as well as dedicated efforts to follow-up cluster strong lensing features with optical/NIR multi-object spectroscopy.  %{\it AvdL comment: The best candidate clusters will probably all well known before LSST 3-year?}



%
%{\it Note from Anja:  Will suggested that strong lensing and/or galaxy velocities can help improve the cluster mass calibration for cluster count cosmology over the default plan of using weak-lensing.  For cluster counts, I remain skeptical on both accounts. Strong lensing is awesome and gives us very precise mass constraints in cluster cores ($\sim 100$~kpc), but requires significant extrapolation to measure masses on the scales of $R_{500} - R_{200}$ ($\sim 1$~Mpc).  Moreover, there is a selection bias towards clusters that are over-concentrated or aligned along the line-of-sight.  While there is simulation work to account for such biases (e.g. Meneghetti et al.), the small scales that strong lensing probes require a thorough understanding of the gas physics at cluster cores, too, which still seems a daunting task.  I DO think strong lensing has a place in precise measurements of cluster density profiles in selected clusters, which is (1) informative for the weak-lensing work, and (2) can be used to probe the processes at cluster cores (both dark matter and cluster astrophysics, see work by Annika Peter).  Regarding the caustic method, I have only seen one blind test done on simulations (Old et al. 2014, 2015), in which it did not perform particularly well.  Those works showed that in general, velocity-based methods suffer even larger biases and scatters than weak-lensing masses.  I'm willing to believe that things look better if one has several hundreds of velocities per cluster, but that would be a significant investment of telescope time just to be comparable to the weak-lensing constraints.  I actually think that modified gravity is a much better selling point for such datasets.\\
%I've modified the point on cluster redshifts to focus it on high-redshift clusters.  At $z<0.8$, redMaPPer/redMaGiC already perform extraordinarily well, and any overlap with spectroscopic samples is probably enough to further confirm that to z~1.2.\\
%I'm not sure whether Will meant cosmology with the most-massive-cluster test (Mortonson et al. 2013)?  I think this outside the LSST realm, because the most massive high-redshift clusters will first be found by the SZ surveys.  Also, this test requires multiple mass measurement, and I think deep weak-lensing data for such candidates will be gathered before LSST 10-year depth is complete.}


%% ----------------------------------------------------------------------
%
%\subsection{Cluster Mass Function: Spectroscopic Surveying in Cluster Fields}
%\label{sec:sl:clusters}
%{\it Will Dawson and others}
%
%{\it Note from Will: While I have done a fair amount of cluster weak lensing and
%spectroscopy work, this has been focused on dark matter science, I am not
%actually an expert in cluster mass function cosmological constraints. That said
%I have drafted some text that I hope will describe the methods and needed
%capability (or capabilities) in enough detail that someone could determine
%whether they overlap with other needs. I am also includeing all spectroscopic
%cluster studies and those limited to strong lensing; we can move the text later
%if we like.}
%
%{\it Basic Premis}: Cosmology is sensitive to the abundance of rare massive
%clusters observed out to and beyond $z\sim 1$. For example, Takada \& Spergel
%(2014) have shown that a characterization of the most massive clusters can
%improve the constraining power of future cosmic shear surveys (e.g., LSST, and
%WFIRST) by approximately a factor of two, provided the cluster masses and
%redshifts are accurately constrained.
%
%Spectroscopy of cluster galaxies can improve LSST cluster mass function
%constraints in a number of ways:
%\begin{itemize}
%  \item {\it Redshifts of multiply imaged background galaxies}: Combined strong
%  and weak lensing measurements of clusters can improve both the cluster mass
%  and concentration constraints by approximately a factor of two (Umetsu et
%  al.~2010). So it is possible to improve the LSST cluster weak lensing
%  constraints by combining strong lensing measurement of the strongly lensed
%  background galaxies (preseumably identified from space-based observations,
%  such as HST or WFIRST, but potentially from LSST alone). However, redshifts of
%  the multiply lensed galaxies are needed in order to properly constrain the
%  cluster mass distribution. There has been some work recently by Keren Sharon's
%  group on the number of multiply lensed galaxy redshifts that are necessary for
%  a given lens model accuracy. Most of the multiply lensed galaxies are faint $R
%  \gtrsim 23$ and require large diameter telescopes.
%  \item {\it Accurate cluster redshifts}: For cosmological constraints it is
%  necessary to know both the mass and redshift of the clusters. While it is
%  possible to obtain better than average photometric redshifts of clusters due
%  to their being hundreds to thousands of galaxies to beat down the Poission
%  photo-z noise and the tight red sequence color relation, constaints can be
%  improved (how much) with accurate redshifts from spectroscopy. It is possible
%  to accomplish this with one to a few redshifts of the brightest cluster
%  galaxies. The instrument requirements are pretty relaxed for this science
%  since on only needs a few redshifts per pointing of rather bright galaxies.
%  \item {\it Cluster galaxy velocities as a probe of the cluster potential}: The
%  spectroscopic caustic method (Kaiser 1987; Regos \& Geller 1989) is the only
%  other successful stand alone method, in addition to weak lenising, of probing
%  the mass density profile to comparably large cluster radii. This is important
%  since joint CMB+SZE constraints indicate that SZE-derived {\em Planck} masses
%  underestimate the true $M_{500}$ masses by $b=1-\langle
%  M_{Planck}/M_\mathrm{true}\rangle\sim 43\%$ $\sim43\%$ --- far from the $\sim
%  20\%$ expected due to deviations from the influence of feedback, non-thermal
%  pressure, and cluster shapes (Battaglia+ 2012). Of even greater concern is
%  that according to WL studies (Sereno \& Ettori 2015), the bias appears to vary
%  with mass. SZE mass estimates are calibrated in part based on X-ray
%  temperatures with spectroscopy that is only reliable out to $r_{2500}\sim
%  0.25r_\mathrm{vir}$. Recent WL analyses of the {\em Planck} sample, that probe
%  the mass distribution to larger radii, yield $1-\langle
%  M_{Planck}/M_\mathrm{WL}\rangle\sim 30\%$ at the $\sim 10\%$ uncertainty level
%  (von der Linden+ 2014; Hoekstra+ 2015), somewhat relaxing the {\em Planck}
%  tension. Hinting at the importance of having probes of the cluster potential
%  out to large radii ($\sim$ the virial radius). To realize this science would
%  require more than a hundred spectroscopic redshifts per cluster, with a
%  moderate resolution spectrograph. A highly multiplexed moderate resolution
%  spectrograph, that can simulatinously target objects within $\lesssim
%  5$\,arcsecons of one another (due to the dense cluster environment). I (wd) am
%  not sure how many clusters would need to be surveyed to make a significant
%  improvement to planned LSST cluster cosmology.
%  \item {\it Redshift space distortions}: This is a probe of modified gravity,
%  and similar to the spectroscopic caustic method requires hundreds of
%  spectroscpic redshifts of cluster galaxies. The spectroscopic demands are the
%  same as the caustic method as well. I (wd) am not sure how many clusters we
%  would need to make this measurement for.
%\end{itemize}
%
%\ldots
%
%% ======================================================================
